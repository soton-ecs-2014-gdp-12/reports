\documentclass[12pt,a4paper]{article}
\usepackage{pgfgantt}
\usepackage{pdflscape}
\usepackage{url}
\usepackage{standalone}
\usepackage[hidelinks]{hyperref}
\usepackage[intoc]{nomencl}
\usepackage{caption}
\usepackage{parskip}
\usepackage{menukeys}

\usepackage{pdfpages}
\usepackage{todonotes}

\usepackage[top=2.4cm, bottom=2.4cm, inner=3.5cm, outer=2.4cm]{geometry}

\usepackage{microtype}

\begin{document}
\pagenumbering{gobble}

\begin{Large}
\textbf{School of Electronics And Computer Science
}\end{Large}\\

\begin{Large}
ELEC6200 MEng Group Design Project, Group 12
\end{Large}\\

\begin{center}

{\href{mailto:cb15g11@soton.ac.uk}{Christopher Baines}}\\
{\href{mailto:sb21g11@soton.ac.uk}{Samuel Bennett}}\\
{\href{mailto:hc13g11@soton.ac.uk}{Harry Cutts}}\\
{\href{mailto:cjh1e11@soton.ac.uk}{Christopher Hewett}}\\
{\href{mailto:ml26g11@soton.ac.uk}{Maria Lynch}}

\end{center}

\tableofcontents

\section{Customer Deliverables}


It has been agreed during the weekly customer meeting on 1st December 2014 that the following deliverables shall be produced as part of the final project:

\begin{itemize}
\item Videogular Questions: a plugin for adding polls and questions to videos
	\begin{itemize}
	\item Source code, under the MIT licence
	\item An example proof-of-concept website
	\end{itemize}
	
\item Videogular Cuepoints: a plugin which displays informational marks on the scrub bar of a video
	\begin{itemize}
	\item Source code, under the MIT licence
	\item Demonstration of usage in Videogular Questions example site
	\end{itemize}
	
\item Videogular Heatmap: a plugin which displays heat map information on the scrub bar of a video
	\begin{itemize}
	\item Source code, under the MIT licence
	\end{itemize}
	
\item Videogular Analytics: a plugin for reporting of events within the Videogular player
	\begin{itemize}
	\item Source code, under the MIT licence
	\item An API specification for Videogular Analytics
	\item An example site showing basic usage of the analytics data collected from the plugin
	\end{itemize}
	
\item Authoring tool: a Web application to produce the JavaScript file to be used with Videogular Questions
	\begin{itemize}
	\item Source code, under the MIT licence
	\end{itemize}
\end{itemize}

Links to \url{http://kanga-cb15g11.ecs.soton.ac.uk} will require access to the ECS network. Accessing this website from a lab machine or directly networked computer will work, otherwise the ECS VPN will be required.

\section{Videogular Cuepoints}

The plugin source code is available to download from our repository located at

\url{https://github.com/soton-ecs-2014-gdp-12/videogular-cuepoints}

Usage instructions can be found in the README file in that repository, and an example can be found in the Videogular Questions example.

\section{Videogular Heatmap}

The plugin source code is available to download from our repository located at

\url{https://github.com/soton-ecs-2014-gdp-12/videogular-heatmap}

Usage instructions can be found in the README file in that repository.

\section{Videogular Analytics}

The plugin source code is available to download from our repository located at

\url{https://github.com/soton-ecs-2014-gdp-12/videogular-analytics}

The example analytics back end has been set up as a demo at

\url{http://kanga-cb15g11.ecs.soton.ac.uk:5001/}

\subsection{Feature Walkthrough}

The example that we have written for the Videogular Analytics plugin has a storage and replay facility. You can store a list of events received with a JSON file and replay them later. This allows you to post-process events as needed.

To demonstrate the features of the analytics back end, we have loaded some sample data into the example website.

However, since this is only an example of how you can use the analytics, not a deliverable product, it is not fully featured. Further processing methods could be implemented in future, but all current features work.

The following demonstration features have been implemented:

\begin{itemize}
\item Events Log page
	\begin{itemize}
	\item The top of the page lists all events received, including the event type, the time received, and the data held by the events.
	\item The bottom of the page shows the sections of the video that each user has watched. The UUID is used to show different users, but does not identify a user in any way.
	\end{itemize}

\item Per user statistics
	\begin{itemize}
	\item The watched video segments are processed to show how much of the video each user has viewed.
	\item The data is also processed to show which how many times each segment has been watched. There is a known bug with Videogular reporting odd times, causing this statistic to display incorrectly. We have been investigating this, but the data demonstrates a processing possibility. This is the numerical form of the data displayed using the heatmap.
	\end{itemize}

\item Results page
	\begin{itemize}
	\item This shows the results of each question, and the correct answer where possible.
	\end{itemize}

\item Results Correlation
	\begin{itemize}
	\item The ``\% watched by correct answers'' and ``Time watched by correct answers'' graphs show two ways you could represent the data in a different way.
	\item Both of these examples use random data but the majority of the code for producing the graphs has been written. We have abstracted the scatter graph creation code so that arbitrary data sets can be given to the scatter plots.
	\end{itemize}

\end{itemize}

\subsection{Analytics emitted events}

The Videogular Analytics plugin emits a number of events which can be sent to a Web server for collection. The table below describes these events.

The plugin is designed so that events can easily be added by sending events to the `analytics' channel. These are automatically collected and prepared for sending to the given Web server.

\begin{tabular}{p{3.2cm} p{7cm} p{4cm}}

\textbf{Event name} & \textbf{Emitted when} & \textbf{Expected Payload} \\ 
\hline 
show\_question & a question is shown to the user & All of the associated question data \\ 
\hline 
end\_question & the annotation being shown has finished & None \\ 
\hline 
show\_results & there are results that will be shown & The results data being shown \\ 
\hline 
submitted\_question & the user submits a question & The chosen question response \\ 
\hline 
skipped\_question & the user skips a question & None \\ 
\hline 
continue\_question & a results page is closed by pressing the continue button & None \\ 
\hline 
play & the video starts to play & The time the video plays from \\ 
\hline 
pause & the video is paused & The time the video pauses at \\ 
\hline 
stop & the video is stopped & The time the video was stopped at \\
\end{tabular} 


\section{Authoring tool}

The plugin source code is available to download from our repository located at 

\url{https://github.com/soton-ecs-2014-gdp-12/authoring-tool}

The authoring tool has been set up as a demo at

\url{http://kanga-cb15g11.ecs.soton.ac.uk:5002/app/#/authoring}

\subsection{Features}

The authoring tool is being delivered as a working prototype and allows you to create a JavaScript file for the Videogular Questions plugin.

Videogular Questions provides a preview of the questions being created. There will be minor display changes when going from the authoring tool to a working example as the user may style the questions freely.

Pressing \keys{Export} will provide you with a JavaScript file which can be used. Pressing \keys{Preview} will reset the video back to the start, load the questions into the preview, and begin playing.

All 5 question types are implemented and, as suggested, the Single choice question (with radio buttons) is merged with the Multiple choice question. If you create a Multiple choice question with one correct answer and allow the user to select a minimum and maximum of 1 answer a Single choice question will be generated.

For demonstration purposes, the authoring tool has been set up with the Caesar Cipher example. Once Videogular's YouTube plugin has been fixed for the current version this will be easier to modify, however this was agreed not to be a priority for our project.

It is possible to generate a number of JavaScript files which produce invalid or odd questions, such as a rating question with a minimum of 0 and a maximum of 0. To more easily allow further improvements to the authoring tool, it has been decided that we shall not limit what the user can create.

In future, the tool could be made `safe' so that no user can create an invalid question. However, leaving the user to create a quiz precisely how they wish, not limiting the large number of options, allows them much more power.

\subsection{Full Feature Walkthrough}

We have designed a number of tests for the authoring tool. Below is the process we have been going through to ensure that the functionality works as expected. After each step the preview button is pressed to confirm that the JavaScript file is correct, and the JavaScript file is validated by checking it manually.

\begin{itemize}
\item A question set is added at a specific time.
\item A single question is added, then tested to ensure it appears.
\item Each question type is tested along with common options.
\item Skipping is tested for one question.
\item Recording responses and viewing them is tested for one question.
\item The interface is reloaded, then a number of question sets and random questions are created and tested, ensuring that they appear at the correct times.
\end{itemize}

\section{Videogular Questions}

The plugin source code is available to download from our repository located at

\url{https://github.com/soton-ecs-2014-gdp-12/videogular-questions}

An example site can be found at

\url{http://kanga-cb15g11.ecs.soton.ac.uk:8000/app/#/simple-example}

\subsection{Features}

\begin{itemize}
\item Single Question provides radio buttons to select at most one answer.
\item Question Multiple provides checkboxes to select one or more results. The min/max values should be shown and \keys{Submit} should be disabled if these requirements are not met.
\item Questions Star should allow you to click on the stars and select one rating. The Min/Max values should be shown and \keys{Submit} should be disabled if these requirements are not met.
\item Questions Text should allow you to enter a textual value before submitting.
\item Question Range should present a slider which allows you to choose a range of values. The min/max values should be shown on the slider.
\end{itemize}

\subsection{Walkthrough}

Each tab should be testing to see the working solution.

The tabs Single Question, Question Multiple, Question Stars, Question Text, Question Range all should show one demonstration of the specific question type.

The simple example should show a very basic example, this will be similar to Single Question.

The poll simple example should show an example poll.

Caesar Example is used to demonstrate a number of features being used. This has Videogular Cuepoints enabled so you will be able to see where the questions will be shown. The second question at the end of the example will have a poll so you will be able to see the results of all those who have answered the question.

\end{document}
