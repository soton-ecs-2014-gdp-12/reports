\documentclass[12pt,a4paper]{article}
\usepackage{pgfgantt}
\usepackage{pdflscape}
\usepackage{url}
\usepackage{standalone}
\usepackage[hidelinks]{hyperref}
\usepackage[intoc]{nomencl}
\usepackage{caption}
\usepackage{parskip}

\usepackage{pdfpages}
\usepackage{todonotes}

\usepackage[top=2.4cm, bottom=2.4cm, inner=3.5cm, outer=2.4cm]{geometry}

\usepackage{microtype}

\begin{document}
\pagenumbering{gobble}

\begin{Large}
\textbf{School of Electronics And Computer Science
}\end{Large}\\

\begin{Large}
ELEC6200 MEng Group Design Project, Group 12
\end{Large}\\

\begin{center}

{\href{mailto:cb15g11@soton.ac.uk}{Christopher Baines}}\\
{\href{mailto:sb21g11@soton.ac.uk}{Samuel Bennett}}\\
{\href{mailto:hc13g11@soton.ac.uk}{Harry Cutts}}\\
{\href{mailto:cjh1e11@soton.ac.uk}{Christopher Hewett}}\\
{\href{mailto:ml26g11@soton.ac.uk}{Maria Lynch}}

\end{center}

\tableofcontents

\section{Customer Deliverables}


It has been agreed during the weekly customer meeting on 1st December 2014 that the following deliverables shall be produced as part of the final project.

\begin{itemize}
\item Videogular Questions - A plugin which allows a user to add polls and questions on videos
	\begin{itemize}
	\item Source code for the plugin licence under the MIT licence
	\item An example proof-of-concept website
	\end{itemize}
	
\item Videogular Cuepoints - A plugin which displays informational marks on the scrub bar of a video
	\begin{itemize}
	\item Source code for the plugin licence under the MIT licence
	\item Demonstration of usage in Videogular Questions example site
	\end{itemize}
	
\item Videogular Heat Maps - A plugin which displays heat map information on the scrub bar of a video
	\begin{itemize}
	\item Source code for the plugin licence under the MIT licence
	\end{itemize}
	
\item Videogular Analytics - A plugin that allows reporting of events within the Videogular player
	\begin{itemize}
	\item Source code for the plugin licence under the MIT licence
	\item API specification for Videogular Analytics
	\item An example site showing basic usage of the analytics data collected from Videogular Analytics plugin
	\end{itemize}
	
\item Authoring tool - A tool to produce the JavaScript file to be used with Videogular Questions plugin
	\begin{itemize}
	\item Source code for the plugin licence under the MIT licence
	\end{itemize}
\end{itemize}

\section{Videogular Cuepoints}

The plugin source code licensed under the MIT licence is available to download at our repository located at 

\url{https://github.com/soton-ecs-2014-gdp-12/videogular-cuepoints}

The example for this is at end of this document in the Videogular Questions example.

\section{Videogular Heat Maps}

The plugin source code licensed under the MIT licence is available to download at our repository located at 

\url{https://github.com/soton-ecs-2014-gdp-12/videogular-heatmap}

The example for this is in the Videogular Analytics example below.

\section{Videogular Analytics}

The plugin source code licensed under the MIT licence is available to download at our repository located at 

\url{https://github.com/soton-ecs-2014-gdp-12/videogular-analytics}

The example analytics back end has been set up as a demo here

\url{FIXME}

\subsection{Walkthrough}

The example that we have written for the Videogular Analytics has a storage and replay facility. You can store a list of events received with a JSON file and load them up as needed. This allows you to post-process events as needed.

Therefore we have loaded up the example website with some data already collected to more easily demonstrate the features in the analytics back end.

However since this is only an example of how you can use the analytics and not a deliverable product it is not fully featured. Future work could be done to specialise on the information required but all current features work.

A list of features that have been implemented to demonstrate how the analytics could be used are below

\begin{itemize}
\item Videogular Heatmaps
	\begin{itemize}
	\item Due to the below bug this does not currently work. TODO: try and get this working?
	\end{itemize}

\item Events Log page
	\begin{itemize}
	\item The top of the page lists all events received. This shows some of the possible events received, the time received and the data held by the events.
	\item The bottom of the page shows the combined sections that a user has watched. Here the UUID is used to show different users and does not identify a user in any way keeping it anonymous.
	\end{itemize}
	
\item Per user statistics
	\begin{itemize}
	\item The Per User viewed of total processes the video watched segments to work out exactly what has been viewed and the percentage and time representation.
	\item The data is also processed to show which how much each segment has been watched. There is a known bug with Videogular reporting odd times causing this statistic to display incorrectly. We have been doing investigations into the Videogular player currently but this demonstrates what can be processed. This data is the numerical view of the heatmap.
	\end{itemize}

\item Results page
	\begin{itemize}
	\item This shows the results of each possible question and the correct answer where possible.
	\end{itemize}

\item Results Correlation
	\begin{itemize}
	\item The \% watched by correct answers and Time watched by correct answers graphs show two ways you could possible represent the data in a different way.
	\item Both of these examples use random data but the majority of the code, producing the graphs, has been written. We have abstracted the scatter graph creation code so that arbitrary data sets can be given to the scatter plots. This should reduced much of the required code if this example were to be expanded.
	\end{itemize}

\end{itemize}

\subsection{Analytics emitted events}

The Videogular analytics plugin emits a number of events which can be sent to a webserver for collection. This guide is the list of events that are currently emitted, what and when is emitted.

The plugin is designed so that events can be easily added by sending emitting events to the `analytics' channel. These are automatically collected and prepared for sending to the given webserver.

\begin{tabular}{p{3.2cm} p{7cm} p{4cm}}

\textbf{Event name} & \textbf{Event emitted reason} & \textbf{Expected Payload} \\ 
\hline 
show\_question & This occurs when a question is shown to the user & All of the associated question data will be sent \\ 
\hline 
end\_question & This occurs when the annotation being shown has finished & No additional data will be sent \\ 
\hline 
show\_results & This occurs when there are results that will be shown & The results data being shown will be sent \\ 
\hline 
submitted\_question & This occurs when the user has submitted a question & The chosen question response \\ 
\hline 
skipped\_question & This occurs when a question is skipped & none \\ 
\hline 
continue\_question & This occurs when a results page is closed by pressing the continue button & none \\ 
\hline 
play & This occurs when the video starts to play & Timestamp the video plays from \\ 
\hline 
pause & This occurs when the video is paused & Timestamp the video pauses at \\ 
\hline 
stop & This occurs when the video is stopped & Timestamp the video was stopped at \\
\end{tabular} 


\section{Authoring tool}

The plugin source code licensed under the MIT licence is available to download at our repository located at 

\url{https://github.com/soton-ecs-2014-gdp-12/authoring-tool}

The authoring tool has been set up as a demo here

\url{FIXME}

\subsection{Expected Features}



\subsection{User Story Walkthrough}

\section{Videogular Questions}

The plugin source code licensed under the MIT licence is available to download at our repository located at 

\url{https://github.com/soton-ecs-2014-gdp-12/videogular-questions}

This should be set up as a demo here

\url{FIXME}

\subsection{Expected Features}

\begin{itemize}
\item Single questions provide radio buttons to select at most one question
\item Multiple questions provide checkboxes to select one or more results. The min/max values should be shown and submit should not be able to be pressed if these requirements are not met.
\item Questions Star should allow you to click on the stars and select one rating. The Min/Max values should be shown and submit should not be able to be pressed if these requirements are not met.
\item Questions Text should allow you to enter a textual value before submitting.
\item Question Range should present a slider which allows you to choose a range of values. The min/max values should be shown and the slider should have these values
\end{itemize}

\subsection{Walkthrough}

Each tab should be testing to see the working solution.

The tabs Single Question, Question Multiple, Question Stars, Question Text, Question Range all should show one demonstration of the specific question type.

The simple example should show a very basic example, this will be similar to Single Question.

The poll simple example should show an example 

Caesar Example is used to demonstrate a number of features being used. This has Videogular Cuepoints enabled so you will be able to see where the questions are shown. The second question at the end of the example will have a poll so you will be able to see the results of all those who have answered the question.

\end{document}
