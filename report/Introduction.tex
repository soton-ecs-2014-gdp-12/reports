%% ----------------------------------------------------------------
%% Introduction.tex
%% ----------------------------------------------------------------
\chapter{Introduction}
\label{Chapter:Introduction}

\begin{preamble}
	\preamblequote{Without annotations a video is like a book without a contents.}{Professor Mike Wald on Synote}
\end{preamble}

\section{Problem}
\label{Section:Problem}

Videos combined with quizzes and polls are used for educational purposes in many settings, including \glspl{MOOC} and university lectures. For the authors of these resources, this often involves splitting videos up into sections with a poll or quiz after each section to gauge understanding. This approach requires the author to have video editing skills, and can be time-consuming. It would be useful if quizzes could be created and included directly into videos without the need for video editing. The resulting media could be analysed to discover how best to present information for learning.

\section{Goals and Scope}
\label{Section:Goals and Scope}

This project will be split into three main sections:

\begin{description}[%
  before={\setcounter{descriptcount}{0}},%
  ,font=\bfseries\stepcounter{descriptcount}\thedescriptcount~]
	\item[Quiz Authoring Tool] \hfill \\
	The authoring tool (designed with accessibility in mind) will allow users to specify:

	\begin{itemize}
		\item sets of questions and polls to appear in the video,
		\item the time at which question sets should appear, and
		\item actions to be taken when questions are answered (e.g. skip back in the video if the answer is incorrect).
	\end{itemize}

	The export feature will produce a \gls{DF} that contains all of the above information that is compatible with the video player and its plugins.

	\item[Questions Overlay] \hfill \\
	A library will be implemented to allow quizzes and polls to be overlaid on Web videos, playable in the major browsers (Mozilla Firefox, Microsoft Internet Explorer and Google Chrome) on different platforms (PC, Mac OS X and Android). These videos could be any of the main video formats (MP4, WebM, OGG). The library will read the \gls{DF} and show the questions on the overlay at the appropriate time. Accessibility considerations for the overlaid videos will be investigated.

	\item[Video and Quiz Analytics] \hfill \\
	Metrics of user behaviour will be recorded and methods of displaying these to the author will be explored.
\end{description}
