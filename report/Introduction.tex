%% ----------------------------------------------------------------
%% Introduction.tex
%% ---------------------------------------------------------------- 
\chapter{Introduction} 
\label{Chapter:Introduction}

\begin{preamble}
	\preamblequote{Without annotations a video is like a book without a contents.}{Professor Mike Wald on Synote}
\end{preamble}

\section{Problem} 
\label{Section:Problem}

Videos combined with quizzes and polls are often used for educational purposes in many applications including \glspl{MOOC} and university lectures. For the authors of these resources this often involves splitting videos up into sections with a poll/quiz after each section to gauge understanding. Using this approach requires the author to have video editing skills in order to insert the quizzes. It would be useful if quizzes could be saved and included directly into videos without the need to edit them. The new media created could be analysed to make discoveries about how best to present information to viewers so that they can take the most information in. 

\section{Goals and Scope} 
\label{Section:Goals and Scope}
This project will be split into three main sections:
\begin{description}[%
  before={\setcounter{descriptcount}{0}},%
  ,font=\bfseries\stepcounter{descriptcount}\thedescriptcount~]
\item[Quiz Authoring Tool] \hfill \\
The authoring tool (designed with accessibility in mind) will allow users to:
\begin{itemize}
\item Specify sets of questions and polls to appear in the video
\item The time at which question sets should appear
\item Actions that should be taken when questions are answered (e.g. skip back to a certain point if the answer is incorrect)
\end{itemize}

The export feature will produce a \gls{DF} compatible with the video player that contains all of the above information.

\item[Overlaid Video Player] \hfill \\
A library will be implemented which allows quizzes and polls to be overlaid on web videos, playable in the main browsers (Firefox, IE and Chrome) on different platforms (PC, Mac and Android). These videos could be any of the main video formats (MP4, WebM, OGG). The \gls{DF} specifying the questions will be interpreted by the library and shown in the overlay. Accessibility considerations for the overlaid videos will be investigated.
\item[Video and Quiz Analytics] \hfill \\
Metrics of user behaviour will be recorded and methods of displaying these to the author will be explored.
\end{description}