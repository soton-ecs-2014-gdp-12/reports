%% ----------------------------------------------------------------
%% AppendixA.tex
%% ---------------------------------------------------------------- 
\chapter{Stuff} \label{Chapter:Stuff}
The following gets in the way of the text....

\chapter{Skills Audit Results} \label{Chapter:Skills Audit Results}

A skills audit of the groups members was performed and 

Project Management

Planning 
Progress tracking 
Time management (Gantt chart) 
Contingency planning (Risk Assessment) 

Technical

Technical Research
Analysis
Architecture Design
API Design
HCI / Interface Design
Implementation
	Client Side
		HTML
		CSS
		Javascript
			AngualrJS
	Server Side
		Python
			Flask
		Javascript
			Node
Evaluation

Communication

Technical Writing
	Software Documentation
	Report
Academic Reaserch
Presentation
Critical and comparative evaluation 
Reflection 

\chapter{Deployment of the Poll Server} \label{Chapter:Deployment Poll Server}

To deploy the poll server on using apache first you need to clone the repository.

Then the apache httpd.conf file needs to be configured by adding the following entry somewhere in the file:

\begin{lstlisting}
<VirtualHost <hostname>:80>
	ServerName <hostname>
	WSGIDaemonProcess poll_server user=<user> group=<group> threads=5
	WSGIScriptAlias / /<location>/poll_server.wsgi
	ErrorLog logs/poll_server-error_log
	CustomLog logs/poll_server-access_log common

	<Directory <location>>
		WSGIProcessGroup poll_server
		WSGIApplicationGroup %{GLOBAL}
		Order deny,allow
		Allow from all
	</Directory>
</VirtualHost>
\end{lstlisting}

Parameters needing changes:

\begin{itemize}
\item \textless hostname\textgreater is the hostname of the server. e.g. website.domain.net
\item \textless location\textgreater is the location of the sourcecode on the server. e.g. /var/www/poll\_server/
\item \textless user\textgreater is the user you want the script to run under, by default apache
\item \textless group\textgreater is the group you want the script to run under, by default apache
\item The `ErrorLog` and `CustomLog` parameters can be changed to any location

\end{itemize}

Details are also available in the the README.md file in the repository.

\chapter{Deployment of the Analytics Server} \label{Chapter:Deployment Analytics Server}

Apache cannot directly run node.js webservices therefore it needs to be ran by node itself.

This can be run by using `npm start` which will run the node server. This needs to be running on the server all the time you want the server to be accessible.
A web search will return details of how to turn a node.js webserver into a service however it can just be run from the command line using screen or a similar program.

You can configure apache to proxy any connections to the node server. This is how we suggest integrating an apache server with node.

To do this mod\_proxy needs to be installed, web searches will find guides to install this for your chosen operating system.

Then the apache httpd.conf file needs to be configured by adding the following entry somewhere in the file:

\begin{lstlisting}
<VirtualHost <hostname>:80>
	ServerName <hostname>
	ErrorLog logs/analytics-error_log
	CustomLog logs/analytics-access_log common

	ProxyRequests off
	
	<Proxy *>
		Order deny,allow
		Allow from all
	</Proxy>

	<Location />
		ProxyPass <analytics address>
		ProxyPassReverse <analytics address>
	</Location>
</VirtualHost>
\end{lstlisting}

Parameters needing changes:

\begin{itemize}
\item \textless hostname\textgreater is the hostname of the server. e.g. hostname.domain.com
\item \textless analytics address\textgreater is the URL for the analytics server. This is displayed when npm start is run. by default this is `http://localhost:5001/`
\item The `ErrorLog` and `CustomLog` parameters can be changed to any location
\end{itemize}

Details are also available in the the README.md file in the repository.