%% ----------------------------------------------------------------
%% Analytics.tex
%% ----------------------------------------------------------------

\chapter{Analytics} \label{Chapter: Analytics}

\chapterpreamble{``I never guess. It is a capital mistake to theorize before one has data. Insensibly one begins to twist facts to suit theories, instead of theories to suit facts.'' \\ - \small{Sir Arthur Conan Doyle, Author of Sherlock Holmes stories}}

\section{Introduction}

For the framework to fulfill the requirements, allowing for analysis of how users are interacting with the system is crucial. This plugin allows for getting the necessary data from each users device to a server where it can be aggregated and processed.

\section{Design}

The analytics functionality is separate from the main vgQuestions plugin as not all users of the core functionality will also want to perform analytics.

Implementing this as a separate plugin means that its not dependant on any of the other plugins written as a part of this project.

This plugin needs to provide a api such that other plugins can send data to a set of analytics servers. This should be done in such a way that does not prevent the operation of the client plugin if the analytic plugin is not available.

\subsection{Analytics emitted events}

The Videogular Analytics plugin emits a number of events which can be sent to a Web server for collection. The table below describes these events.

The plugin is designed so that events can easily be added by sending events to the `analytics' channel. These are automatically collected and prepared for sending to the given Web server.

\begin{tabular}{p{3.2cm} p{6cm} p{4cm}}

\textbf{Event name} & \textbf{Emitted when} & \textbf{Expected Payload} \\
\hline
show\_question & a question is shown to the user & All of the associated question data \\
\hline
end\_question & the annotation being shown has finished & None \\
\hline
show\_results & there are results that will be shown & The results data being shown \\
\hline
submitted\_question & the user submits a question & The chosen question response \\
\hline
skipped\_question & the user skips a question & None \\
\hline
continue\_question & a results page is closed by pressing the continue button & None \\
\hline
play & the video starts to play & The time the video plays from \\
\hline
pause & the video is paused & The time the video pauses at \\
\hline
stop & the video is stopped & The time the video was stopped at \\
\end{tabular}

\section{Implementation}

The plugin itself watches videogular, and triggers events when videogular's state changes. Other plugins can report events by sending a event to angular with the name ``analytics''.

The plugin will then forward all the events to all of the servers.

If the plugin is not in use, then any events reported by other plugins will simply not be processed, as there will be nothing listing for the events called ``analytics''.

