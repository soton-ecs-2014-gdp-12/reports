%% ----------------------------------------------------------------
%% Analytics.tex
%% ----------------------------------------------------------------

\chapter{Analytics} \label{Chapter: Analytics}

\begin{preamble}
	\preamblequote{I never guess. It is a capital mistake to theorize before one has data. Insensibly one begins to twist facts to suit theories, instead of theories to suit facts.}{Sir Arthur Conan Doyle, Author of Sherlock Holmes stories}
\end{preamble}

\section{Introduction}

One of the main requirements states  that video and question events should be emitted from the Videogular player (\cref{Req:Analytics events}. We decided that instead of putting analytics events directly into the \gls{vgQuestions} plugin we would make a new plugin to handle analytics.

This plugin will monitor how the user uses the videogular video player and the vgQuestions plugin and store events related to this. If a server is configured this will send the users data to a server where it can be aggregated and processed.

\section{Design}

The reason for designing the plugin to be separate from the main vgQuestions plugin is because not all users of the core functionality will also want to perform analytics. This means that for users that are not interested their will be no overhead of collecting the data. This can be important on mobile devices that have smaller processing power available.

In addition, implementing this as a separate plugin means that it is not dependant on any of the other plugins and therefore fulfils \cref{Req:Standalone}.

Since the plugin is the published deliverable it will be used with an external server to capture the data. This will be able to aggregates this and store it in some format. It is configured to post data via \gls{REST} calls (\cref{Req:Server architecture}) to a server defined in the analytics configuration file.

There will be a published \gls{API} to be used with the analytics plugin that will detail all events will be published form the plugin. A well documented API is important (\cref{Req:Documentation}) so that developers can easily create a service that collects and uses the data.

In the event that the server is unable to be contacted it should queue the results until it comes back online. Once a request fails to be sent the plugin should try sending all queued events at an interval until the server comes back online or the application is closed. If the application is closed before the server comes back online the data will be lost.

\subsection{Analytics emitted events}

\begin{figure}
\centering
\begin{sequencediagram}
  \newinst[0.5]{q}{vgQuestions plugin}
  \newinst[0.3]{c}{analytics channel}
  \newinst[0.3]{a}{vgAnalytics}
  \newinst[1.4]{r}{REST Service}

  \begin{sdblock}{vgQuestions}{}
  	\begin{call}
  	  {q}{Video Events}
  	  {c}{}
  	\end{call}
  \end{sdblock}

  \begin{sdblock}{vgAnalytics}{}

  \mess{c}{Video Events}{a}

  \stepcounter{seqlevel}  
    \begin{call}
    {a}{Analytic event}
    {r}{Acknowledgement}
  \end{call}
  
  \end{sdblock}

\end{sequencediagram}
\caption{Sequence diagram showing the process of events being emitted by vgQuestions plugin and sent to the REST service by the vgAnalytics plugin}
\label{Figure:sequence_diagram_vgAnalytics}
\end{figure}

The Videogular Analytics plugin emits a number of events which can be sent to a Web server for collection. \autoref{Table:analytics_api} describes all events emitted from the analytics plugin.

The plugin is designed so that events can easily be added by sending events to the \pre{analytics} channel. These are automatically collected and prepared for sending to the given Web server.

\begin{tabular}{p{3.2cm} p{6cm} p{4cm}}
\caption{\label{Table:analytics_api}API of the emitted analytics events and their data payload}
\textbf{Event name} & \textbf{Emitted when} & \textbf{Expected Payload} \\
\hline
show\_question & a question is shown to the user & All of the associated question data \\
\hline
end\_question & the annotation being shown has finished & None \\
\hline
show\_results & there are results that will be shown & The results data being shown \\
\hline
submitted\_question & the user submits a question & The chosen question response \\
\hline
skipped\_question & the user skips a question & None \\
\hline
continue\_question & a results page is closed by pressing the continue button & None \\
\hline
play & the video starts to play & The time the video plays from \\
\hline
pause & the video is paused & The time the video pauses at \\
\hline
stop & the video is stopped & The time the video was stopped at \\
\end{tabular}

\section{Implementation}

The plugin itself watches videogular, and triggers events when videogular's state changes. Other plugins can report events by sending a event to angular with the name ``analytics''.

The plugin will then forward all the events to all of the servers. It does this in a standard format that make it easy to process all the events, and just select the ones that are required for the analysis that is being performed.

If the plugin is not in use, then any events reported by other plugins will simply not be processed, as there will be nothing listing for the events called ``analytics''.

