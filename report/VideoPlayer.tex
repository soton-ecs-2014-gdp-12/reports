%% ----------------------------------------------------------------
%% VideoPlayer.tex
%% ---------------------------------------------------------------- 
\chapter{Video Player} 
\label{Chapter:Video Player}

\section{Accessibility} 
\label{Section:Accessibility}
Videogular has the option to display HTML controls for the video player, replacing the browser's built-in controls. However, when the project began there was no way for the user to access these controls without a mouse.The main problem was that the controls did not identify themselves as interactive elements, and so could not be 'focused' (selected with the tab key). This also prevented them from being picked up by a screen reader.

Some online video players allow the player as a whole to be focused, and then controlled with various keyboard shortcuts. The problem with this method is that accessibility aids, such as screen readers, will not have information on these controls, and so users of those technologies will have to be instructed. It is also not accessible to users of a clicker, who can only press one key and so require individually focusable controls.

Instead, the controls were made individually focusable by using semantic markup where possible, and ARIA attributes otherwise. When a control is in focus, a visual indicator is displayed to highlight it. Buttons can then be pressed using the space bar (\fref{Figure:Accessibility/Screenshots/Button}); the scrub bar can be moved using the left and right arrow keys (\fref{Figure:Accessibility/Screenshots/ScrubBar}); and the volume can be adjusted using the up and down arrow keys when the mute button is focused (\fref{Figure:Accessibility/Screenshots/Volume}).

\begin{figure}
	\begin{subfigure}[]{\textwidth}
		\includegraphics[width=\textwidth]{accessibility/button}
		\caption{The play button when focused. In this state, pressing the space bar plays or pauses the video.}
		\label{Figure:Accessibility/Screenshots/Button}
	\end{subfigure}
	\begin{subfigure}[]{\textwidth}
		\includegraphics[width=\textwidth]{accessibility/scrub-bar}
		\caption{The scrub bar when focused. In this state, pressing the left or right arrow keys skips backwards or forwards in the video.}
		\label{Figure:Accessibility/Screenshots/ScrubBar}
	\end{subfigure}
	\begin{subfigure}[]{\textwidth}
		\includegraphics[width=\textwidth]{accessibility/volume}
		\caption{The mute button and volume control when the mute button is focused. In this state, pressing the space bar toggles mute, and the up and down arrow keys change the volume.}
		\label{Figure:Accessibility/Screenshots/Volume}
	\end{subfigure}
	\caption{Screenshots of the accessibility improvements made to Videogular's HTML controls.}
	\label{Figure:Accessibility/Screenshots}
\end{figure}

This solution is not completely clicker-accessible, but could be made so by introducing buttons to skip forwards or backwards, and to raise or lower the volume. It is a significant improvement on the original state, as keyboard users can now use the player. The improvements were submitted to the Videogular project\footnote{\url{https://github.com/2fdevs/videogular/pull/108}}, and are now part of Videogular 0.6.1.

\section{Compatibility} 
\label{Section:Compatibility}

One of the main focuses of this part of the project was to make sure the application was compatible with a range of platforms, operating systems and devices. The first challenge was to ensure that the player and overlay displayed correctly at a variety of resolutions and screen/window sizes. 

Apple iPhones provided a problem as iOS forced the video to be played in fullscreen mode. This is a design choice and there is no way to play the video inline, directives to do this are ignored by the browser.

This meant that the video paused at the correct time but as the video is in the native player the overlay was not visible unless the player is quit. There is no way of notifying the user that this needs to be done. A possible solution involves telling the user to quit out of the video when it is paused so they may answer the question. However this is not very user friendly. iOS software on Apple tablets however does not have this limitation and the video will play inline correctly as the device is large enough.

Android phones correctly interpret the inline directive and will properly play this. This ensures that the overlay is correctly applied and usable. This has been the primary focus on testing on mobile devices as it represents a very large portion of the market.

The overlay which displays the questions that the user is asked appears above the video and obstructs the view of the video. Placing the overlay in the middle of the screen requires a large amount of CSS to correctly locate it on the screen for mobile and smaller sized devices. There are still some issues with the popup location on some mobile devices because the browser reports one screen size and uses another. Until these devices are compliant with the specification it is time consuming to write code to fix each individual case.

The best user experience with this application is using it on a desktop that has an updated browser. This will definitely work and display as intended. As the screen size of the device gets smaller the user experience is significantly reduced as the video and poll overlay will be much smaller than recommended. To support a number of browsers and operating systems a range of video formats are supplied to the videogular plugin so that the browser may pick one that can be played.