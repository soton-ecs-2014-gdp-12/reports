\chapter{Code Fragments} \label{Chapter:Code Fragments}

\begin{preamble}
	An appendix with large code fragments not suitable for inclusion in the main body of the text.
	\preamblequote{Measuring programming progress by lines of code is like measuring aircraft building progress by weight.}{Bill Gates}
\end{preamble}

\section{Videogular Examples Code Fragments}

These code fragments have been taken from the videogular Examples repository \autoref{Section:Repo_videogular-questions-example}.

\subsection{Videogular Questions Example Question Definition File}

\autoref{code:questionworker} is an example of the Question definition file used by the VideogularQuestions repository.

\begin{lstlisting}[language=javascript,caption={Code for loading an annotation},label={code:questionworker} ]
importScripts("../questions-worker.js");

loadAnnotations({
  "first-annotation": {
    // the time that this annotation will show up
    // (in seconds from the start of the video)
    time: 1,
    items: [
      {
        id: "first-question",
        type: "single",
        question: "What is the moon made of?",
        options: [
          {
            name: "cheese"
          },
          {
            name: "cheeese"
          },
          {
            name: "cheeeeeese"
          }
        ],
        correctAnswer: "cheese"
      },
      {
        id: "check-question",
        type: "single",
        question: "Answer incorrect, do you want to review the video",
        options: [
          {
            name: "Yes"
          },
          {
            name: "No"
          }
        ],
        action: function(questions, video) {
          if (questions.get("check-question").response === "Yes") {
            video.setTime(0);
          }
        },
        condition: function(questions) {
          return questions.get("first-question").isNotCorrect();
        }
      }
    ]
  }
});
\end{lstlisting}

\section{Authoring Tool Code Fragments}

These code fragments have been taken from the authoring tool repository \autoref{Section:Repo_authoring_tool}.

\subsection{Authoring tool rewatch template}

\autoref{code:authoringToolTemplateRewatchSection} is an example of the template used when the user is asked if they wish to review the video.

\begin{lstlisting}[language=javascript,caption={Base template for a question asking if the viewer wishs to review the video section they answered incorrectly},label={code:authoringToolTemplateRewatchSection} ]
{
	id: "incorrect-QUESTION_ID",
	type: "single",
	question: "Answer incorrect, do you want to review the video",
	options: [
		{
			name: "Yes"
		},
		{
			name: "No"
		}
	],
	action: function(questions, video) {
		var question = questions.get("QUESTION_ID");
		if (question.response !== question.correctAnswer) {
			video.setTime(TIME);
		}
	},
	condition: function(questions) {
		return questions.get("QUESTION_ID").isNotCorrect();
	}
}
\end{lstlisting}