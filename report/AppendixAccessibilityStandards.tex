%% ----------------------------------------------------------------
%% AppendixAccessibilityStandards.tex
%% ---------------------------------------------------------------- 
\chapter{Accessibility Standards} 
\label{Chapter:Accessibility Standards}
Both of our example user interfaces were tested against the \gls{WCAG} 2.0 Standards\footnote{\url{http://www.w3.org/TR/WCAG20/} (Accessed: 15 Jan 15)} to check for accessibility.

\section{Videogular Questions Example}
\label{Section: Conformance of Videogular Questions Example}
\begin{longtable}{|L{0.6}|L{0.03}|L{0.28}|} 
\caption{\label{table: vqe conformance}Conformance to WCAG 2.0 Guidelines for Videogular Questions Example} \\
\hline \textbf{Standard} & \rot{\textbf{Pass }} & \textbf{Comment}\\ \hhline{|===|} \endhead
\multicolumn{3}{c}{Continues on the next page...} \endfoot
\endlastfoot
\textbf{1.1 Text Alternatives:} Provide text alternatives for any non-text content so that it can be changed into other forms people need, such as large print, braille, speech, symbols or simpler language. & \XSolidBrush & There is no text alternative for the video but as the idea is to integrate this into Synote giving this feature would cause duplication\eoline
\textbf{1.2 Time-based Media:} Provide alternatives for time-based media. & \XSolidBrush & There are no captions for the video content. This is down to the video uploaded by the user.\eoline
\textbf{1.3.1 Info and Relationships:} Information, structure, and relationships conveyed through presentation can be programmatically determined or are available in text. (Level A) & &  \eoline
\textbf{1.3.2 Meaningful Sequence:} When the sequence in which content is presented affects its meaning, a correct reading sequence can be programmatically determined. (Level A) &  &  \eoline
\textbf{1.3.3 Sensory Characteristics:} Instructions provided for understanding and operating content do not rely solely on sensory characteristics of components such as shape, size, visual location, orientation, or sound. (Level A) &  &  \eoline
\textbf{1.4.1 Use of Color:} Color is not used as the only visual means of conveying information, indicating an action, prompting a response, or distinguishing a visual element. (Level A) & \XSolidBrush & The cuepoints are only conveyed by a different colour on the scrub bar\eoline
\textbf{1.4.2 Audio Control:} If any audio on a Web page plays automatically for more than 3 seconds, either a mechanism is available to pause or stop the audio, or a mechanism is available to control audio volume independently from the overall system volume level. (Level A) & & \eoline
\textbf{1.4.3 Contrast (Minimum):} The visual presentation of text and images of text has a contrast ratio of at least 4.5:1, except for the following: (Level AA) 
\begin{itemize}
\item Large Text: Large-scale text and images of large-scale text have a contrast ratio of at least 3:1;
\item Incidental: Text or images of text that are part of an inactive user interface component, that are pure decoration, that are not visible to anyone, or that are part of a picture that contains significant other visual content, have no contrast requirement.
\item  Logotypes: Text that is part of a logo or brand name has no minimum contrast requirement.
\end{itemize}
 & & \eoline
\textbf{1.4.4 Resize text:} Except for captions and images of text, text can be resized without assistive technology up to 200 percent without loss of content or functionality. (Level AA) & & \eoline
\textbf{1.4.5 Images of Text:} If the technologies being used can achieve the visual presentation, text is used to convey information rather than images of text except for the following: (Level AA)
\begin{itemize}
\item Customizable: The image of text can be visually customized to the user's requirements;
\item Essential: A particular presentation of text is essential to the information being conveyed.
\end{itemize}
Note: Logotypes (text that is part of a logo or brand name) are considered essential.
&  & \\ \hhline{|===|}
\textbf{2.1.1 Keyboard: }All functionality of the content is operable through a keyboard interface without requiring specific timings for individual keystrokes, except where the underlying function requires input that depends on the path of the user's movement and not just the endpoints. (Level A) & & \eoline
\textbf{2.1.2 No Keyboard Trap: }If keyboard focus can be moved to a component of the page using a keyboard interface, then focus can be moved away from that component using only a keyboard interface, and, if it requires more than unmodified arrow or tab keys or other standard exit methods, the user is advised of the method for moving focus away. (Level A)  & & \eoline
\textbf{2.1.3 Keyboard (No Exception): }All functionality of the content is operable through a keyboard interface without requiring specific timings for individual keystrokes. (Level AAA)   & & \eoline
\textbf{2.2 Enough Time: }Provide users enough time to read and use content. & \CheckmarkBold & Level AAA as no user interactions are time sensitive \eoline
\textbf{2.3 Seizures: }Do not design content in a way that is known to cause seizures.  & & \eoline
\textbf{2.4.1 Bypass Blocks: }A mechanism is available to bypass blocks of content that are repeated on multiple Web pages. (Level A)  & & \eoline
\textbf{2.4.2 Page Titled:} Web pages have titles that describe topic or purpose. (Level A) & & \eoline
\textbf{2.4.3 Focus Order:} If a Web page can be navigated sequentially and the navigation sequences affect meaning or operation, focusable components receive focus in an order that preserves meaning and operability. (Level A)  & & \eoline
\textbf{2.4.4 Link Purpose (In Context): }The purpose of each link can be determined from the link text alone or from the link text together with its programmatically determined link context, except where the purpose of the link would be ambiguous to users in general. (Level A)   & & \eoline
\textbf{2.4.5 Multiple Ways:} More than one way is available to locate a Web page within a set of Web pages except where the Web Page is the result of, or a step in, a process. (Level AA)  & & \eoline
\textbf{2.4.6 Headings and Labels:} Headings and labels describe topic or purpose. (Level AA)  & & \eoline
\textbf{2.4.7 Focus Visible:} Any keyboard operable user interface has a mode of operation where the keyboard focus indicator is visible. (Level AA)  & & \eoline
\textbf{2.4.8 Location: }Information about the user's location within a set of Web pages is available. (Level AAA)  & & \eoline
\textbf{2.4.9 Link Purpose (Link Only): }A mechanism is available to allow the purpose of each link to be identified from link text alone, except where the purpose of the link would be ambiguous to users in general. (Level AAA)  & & \eoline
\textbf{2.4.10 Section Headings: }Section headings are used to organize the content. (Level AAA)
\begin{itemize}
\item Note 1: "Heading" is used in its general sense and includes titles and other ways to add a heading to different types of content.
\item Note 2: This success criterion covers sections within writing, not user interface components. User Interface components are covered under Success Criterion 4.1.2.
\end{itemize}
& & \\ \hhline{|===|}
\textbf{3.1.1 Language of Page:} The default human language of each Web page can be programmatically determined. (Level A)  & & \eoline
\textbf{3.2.1 On Focus:} When any component receives focus, it does not initiate a change of context. (Level A)  & & \eoline
\textbf{3.2.2 On Input:} Changing the setting of any user interface component does not automatically cause a change of context unless the user has been advised of the behavior before using the component. (Level A)  & & \eoline
\textbf{3.2.3 Consistent Navigation: }Navigational mechanisms that are repeated on multiple Web pages within a set of Web pages occur in the same relative order each time they are repeated, unless a change is initiated by the user. (Level AA)  & & \eoline
\textbf{3.2.4 Consistent Identification: }Components that have the same functionality within a set of Web pages are identified consistently. (Level AA)  & & \eoline
\textbf{3.2.5 Change on Request: }Changes of context are initiated only by user request or a mechanism is available to turn off such changes. (Level AAA) & & \eoline
\textbf{3.3 Input Assistance:} Help users avoid and correct mistakes.  &  & \\ \hhline{|===|}
\textbf{4.1.1 Parsing:} In content implemented using markup languages, elements have complete start and end tags, elements are nested according to their specifications, elements do not contain duplicate attributes, and any IDs are unique, except where the specifications allow these features. (Level A)  & & \eoline
\textbf{4.1.2 Name, Role, Value:} For all user interface components (including but not limited to: form elements, links and components generated by scripts), the name and role can be programmatically determined; states, properties, and values that can be set by the user can be programmatically set; and notification of changes to these items is available to user agents, including assistive technologies. (Level A)  & & \eoline
\end{longtable}

\section{Videogular Analytics Example}
\label{Section: Conformance of Videogular Analytics Example}
\begin{longtable}{|L{0.6}|L{0.03}|L{0.28}|} 
\caption{\label{table: va conformance}Conformance to WCAG 2.0 Guidelines for Videogular Analytics Example} \\
\hline \textbf{Standard} & \rot{\textbf{Pass }} & \textbf{Comment}\\ \hhline{|===|} \endhead
\multicolumn{3}{c}{Continues on the next page...} \endfoot
\endlastfoot
\textbf{1.1 Text Alternatives:} Provide text alternatives for any non-text content so that it can be changed into other forms people need, such as large print, braille, speech, symbols or simpler language. & \XSolidBrush & There is no text alternative for the video.\eoline
\textbf{1.2 Time-based Media:} Provide alternatives for time-based media. & \XSolidBrush & There are no captions for the video content. This is down to the video uploaded by the user.\eoline
\textbf{1.3.1 Info and Relationships:} Information, structure, and relationships conveyed through presentation can be programmatically determined or are available in text. (Level A) & &  \eoline
\textbf{1.3.2 Meaningful Sequence:} When the sequence in which content is presented affects its meaning, a correct reading sequence can be programmatically determined. (Level A) &  &  \eoline
\textbf{1.3.3 Sensory Characteristics:} Instructions provided for understanding and operating content do not rely solely on sensory characteristics of components such as shape, size, visual location, orientation, or sound. (Level A) &  &  \eoline
\textbf{1.4.1 Use of Color:} Color is not used as the only visual means of conveying information, indicating an action, prompting a response, or distinguishing a visual element. (Level A) & \CheckmarkBold & The heat map data is conveyed by a different colour on the scrub bar but also in a table\eoline
\textbf{1.4.2 Audio Control:} If any audio on a Web page plays automatically for more than 3 seconds, either a mechanism is available to pause or stop the audio, or a mechanism is available to control audio volume independently from the overall system volume level. (Level A) & & \eoline
\textbf{1.4.3 Contrast (Minimum):} The visual presentation of text and images of text has a contrast ratio of at least 4.5:1, except for the following: (Level AA) 
\begin{itemize}
\item Large Text: Large-scale text and images of large-scale text have a contrast ratio of at least 3:1;
\item Incidental: Text or images of text that are part of an inactive user interface component, that are pure decoration, that are not visible to anyone, or that are part of a picture that contains significant other visual content, have no contrast requirement.
\item  Logotypes: Text that is part of a logo or brand name has no minimum contrast requirement.
\end{itemize}
 & & \eoline
\textbf{1.4.4 Resize text:} Except for captions and images of text, text can be resized without assistive technology up to 200 percent without loss of content or functionality. (Level AA) & & \eoline
\textbf{1.4.5 Images of Text:} If the technologies being used can achieve the visual presentation, text is used to convey information rather than images of text except for the following: (Level AA)
\begin{itemize}
\item Customizable: The image of text can be visually customized to the user's requirements;
\item Essential: A particular presentation of text is essential to the information being conveyed.
\end{itemize}
Note: Logotypes (text that is part of a logo or brand name) are considered essential.
&  & \\ \hhline{|===|}
\textbf{2.1.1 Keyboard: }All functionality of the content is operable through a keyboard interface without requiring specific timings for individual keystrokes, except where the underlying function requires input that depends on the path of the user's movement and not just the endpoints. (Level A) & & \eoline
\textbf{2.1.2 No Keyboard Trap: }If keyboard focus can be moved to a component of the page using a keyboard interface, then focus can be moved away from that component using only a keyboard interface, and, if it requires more than unmodified arrow or tab keys or other standard exit methods, the user is advised of the method for moving focus away. (Level A)  & & \eoline
\textbf{2.1.3 Keyboard (No Exception): }All functionality of the content is operable through a keyboard interface without requiring specific timings for individual keystrokes. (Level AAA)   & & \eoline
\textbf{2.2 Enough Time: }Provide users enough time to read and use content. & \CheckmarkBold & Level AAA as no user interactions are time sensitive \eoline
\textbf{2.3 Seizures: }Do not design content in a way that is known to cause seizures.  & & \eoline
\textbf{2.4.1 Bypass Blocks: }A mechanism is available to bypass blocks of content that are repeated on multiple Web pages. (Level A)  & & \eoline
\textbf{2.4.2 Page Titled:} Web pages have titles that describe topic or purpose. (Level A) & & \eoline
\textbf{2.4.3 Focus Order:} If a Web page can be navigated sequentially and the navigation sequences affect meaning or operation, focusable components receive focus in an order that preserves meaning and operability. (Level A)  & & \eoline
\textbf{2.4.4 Link Purpose (In Context): }The purpose of each link can be determined from the link text alone or from the link text together with its programmatically determined link context, except where the purpose of the link would be ambiguous to users in general. (Level A)   & & \eoline
\textbf{2.4.5 Multiple Ways:} More than one way is available to locate a Web page within a set of Web pages except where the Web Page is the result of, or a step in, a process. (Level AA)  & & \eoline
\textbf{2.4.6 Headings and Labels:} Headings and labels describe topic or purpose. (Level AA)  & & \eoline
\textbf{2.4.7 Focus Visible:} Any keyboard operable user interface has a mode of operation where the keyboard focus indicator is visible. (Level AA)  & & \eoline
\textbf{2.4.8 Location: }Information about the user's location within a set of Web pages is available. (Level AAA)  & & \eoline
\textbf{2.4.9 Link Purpose (Link Only): }A mechanism is available to allow the purpose of each link to be identified from link text alone, except where the purpose of the link would be ambiguous to users in general. (Level AAA)  & & \eoline
\textbf{2.4.10 Section Headings: }Section headings are used to organize the content. (Level AAA)
\begin{itemize}
\item Note 1: "Heading" is used in its general sense and includes titles and other ways to add a heading to different types of content.
\item Note 2: This success criterion covers sections within writing, not user interface components. User Interface components are covered under Success Criterion 4.1.2.
\end{itemize}
& & \\ \hhline{|===|}
\textbf{3.1.1 Language of Page:} The default human language of each Web page can be programmatically determined. (Level A)  & & \eoline
\textbf{3.2.1 On Focus:} When any component receives focus, it does not initiate a change of context. (Level A)  & & \eoline
\textbf{3.2.2 On Input:} Changing the setting of any user interface component does not automatically cause a change of context unless the user has been advised of the behavior before using the component. (Level A)  & & \eoline
\textbf{3.2.3 Consistent Navigation: }Navigational mechanisms that are repeated on multiple Web pages within a set of Web pages occur in the same relative order each time they are repeated, unless a change is initiated by the user. (Level AA)  & & \eoline
\textbf{3.2.4 Consistent Identification: }Components that have the same functionality within a set of Web pages are identified consistently. (Level AA)  & & \eoline
\textbf{3.2.5 Change on Request: }Changes of context are initiated only by user request or a mechanism is available to turn off such changes. (Level AAA) & & \eoline
\textbf{3.3 Input Assistance:} Help users avoid and correct mistakes.  &  & \\ \hhline{|===|}
\textbf{4.1.1 Parsing:} In content implemented using markup languages, elements have complete start and end tags, elements are nested according to their specifications, elements do not contain duplicate attributes, and any IDs are unique, except where the specifications allow these features. (Level A)  & & \eoline
\textbf{4.1.2 Name, Role, Value:} For all user interface components (including but not limited to: form elements, links and components generated by scripts), the name and role can be programmatically determined; states, properties, and values that can be set by the user can be programmatically set; and notification of changes to these items is available to user agents, including assistive technologies. (Level A)  & & \eoline
\end{longtable}

\section{Authoring Tool}
\label{Section: Conformance of Authoring Tool}
\begin{longtable}{|L{0.6}|L{0.03}|L{0.28}|}
\caption{\label{table: authoring tool conformance}Conformance to WCAG 2.0 Guidelines for the Authoring Tool} \\
\hline \textbf{Standard} & \rot{\textbf{Pass }} & \textbf{Comment}\\ \hhline{|===|} \endhead
\multicolumn{3}{c}{Continues on the next page...} \endfoot
\endlastfoot
\textbf{1.1 Text Alternatives:} Provide text alternatives for any non-text content so that it can be changed into other forms people need, such as large print, braille, speech, symbols or simpler language. & \XSolidBrush & There is no text alternative for the video. As the video is uploaded by the user this should not be an issue.\eoline
\textbf{1.2 Time-based Media:} Provide alternatives for time-based media. & \XSolidBrush & There are no captions for the video content. This is down to the video uploaded by the user.\eoline
\textbf{1.3.1 Info and Relationships:} Information, structure, and relationships conveyed through presentation can be programmatically determined or are available in text. (Level A) & \CheckmarkBold & Sensible linear read order identified by the WAVE toolbar\footnote{\url{https://wave.webaim.org/toolbar/} (Accessed: 15 Jan 15)} \eoline
\textbf{1.3.2 Meaningful Sequence:} When the sequence in which content is presented affects its meaning, a correct reading sequence can be programmatically determined. (Level A)& \CheckmarkBold & Sensible linear read order identified by the WAVE toolbar\eoline
\textbf{1.3.3 Sensory Characteristics:} Instructions provided for understanding and operating content do not rely solely on sensory characteristics of components such as shape, size, visual location, orientation, or sound. (Level A) & \CheckmarkBold & All instructions are in a textual format \eoline
\textbf{1.4.1 Use of Color:} Color is not used as the only visual means of conveying information, indicating an action, prompting a response, or distinguishing a visual element. (Level A) & \XSolidBrush & The heat map data is only conveyed by a change in colour on the scrub bar\eoline
\textbf{1.4.2 Audio Control:} If any audio on a Web page plays automatically for more than 3 seconds, either a mechanism is available to pause or stop the audio, or a mechanism is available to control audio volume independently from the overall system volume level. (Level A) & \CheckmarkBold & Video content is paused by default and has a play/pause button\eoline
\textbf{1.4.3 Contrast (Minimum):} The visual presentation of text and images of text has a contrast ratio of at least 4.5:1, except for the following: (Level AA) 
\begin{itemize}
\item Large Text: Large-scale text and images of large-scale text have a contrast ratio of at least 3:1;
\item Incidental: Text or images of text that are part of an inactive user interface component, that are pure decoration, that are not visible to anyone, or that are part of a picture that contains significant other visual content, have no contrast requirement.
\item  Logotypes: Text that is part of a logo or brand name has no minimum contrast requirement.
\end{itemize}
 & & \eoline
\textbf{1.4.4 Resize text:} Except for captions and images of text, text can be resized without assistive technology up to 200 percent without loss of content or functionality. (Level AA) & & \eoline
\textbf{1.4.5 Images of Text:} If the technologies being used can achieve the visual presentation, text is used to convey information rather than images of text except for the following: (Level AA)
\begin{itemize}
\item Customizable: The image of text can be visually customized to the user's requirements;
\item Essential: A particular presentation of text is essential to the information being conveyed.
\end{itemize}
Note: Logotypes (text that is part of a logo or brand name) are considered essential.
&  & \\ \hhline{|===|}
\textbf{2.1.1 Keyboard: }All functionality of the content is operable through a keyboard interface without requiring specific timings for individual keystrokes, except where the underlying function requires input that depends on the path of the user's movement and not just the endpoints. (Level A) & & \eoline
\textbf{2.1.2 No Keyboard Trap: }If keyboard focus can be moved to a component of the page using a keyboard interface, then focus can be moved away from that component using only a keyboard interface, and, if it requires more than unmodified arrow or tab keys or other standard exit methods, the user is advised of the method for moving focus away. (Level A)  & & \eoline
\textbf{2.1.3 Keyboard (No Exception): }All functionality of the content is operable through a keyboard interface without requiring specific timings for individual keystrokes. (Level AAA)   & & \eoline
\textbf{2.2 Enough Time: }Provide users enough time to read and use content. & \CheckmarkBold & Level AAA as no user interactions are time sensitive \eoline
\textbf{2.3 Seizures: }Do not design content in a way that is known to cause seizures.  & & \eoline
\textbf{2.4.1 Bypass Blocks: }A mechanism is available to bypass blocks of content that are repeated on multiple Web pages. (Level A)  & & \eoline
\textbf{2.4.2 Page Titled:} Web pages have titles that describe topic or purpose. (Level A) & & \eoline
\textbf{2.4.3 Focus Order:} If a Web page can be navigated sequentially and the navigation sequences affect meaning or operation, focusable components receive focus in an order that preserves meaning and operability. (Level A)  & & \eoline
\textbf{2.4.4 Link Purpose (In Context): }The purpose of each link can be determined from the link text alone or from the link text together with its programmatically determined link context, except where the purpose of the link would be ambiguous to users in general. (Level A)   & & \eoline
\textbf{2.4.5 Multiple Ways:} More than one way is available to locate a Web page within a set of Web pages except where the Web Page is the result of, or a step in, a process. (Level AA)  & & \eoline
\textbf{2.4.6 Headings and Labels:} Headings and labels describe topic or purpose. (Level AA)  & & \eoline
\textbf{2.4.7 Focus Visible:} Any keyboard operable user interface has a mode of operation where the keyboard focus indicator is visible. (Level AA)  & & \eoline
\textbf{2.4.8 Location: }Information about the user's location within a set of Web pages is available. (Level AAA)  & & \eoline
\textbf{2.4.9 Link Purpose (Link Only): }A mechanism is available to allow the purpose of each link to be identified from link text alone, except where the purpose of the link would be ambiguous to users in general. (Level AAA)  & & \eoline
\textbf{2.4.10 Section Headings: }Section headings are used to organize the content. (Level AAA)
\begin{itemize}
\item Note 1: "Heading" is used in its general sense and includes titles and other ways to add a heading to different types of content.
\item Note 2: This success criterion covers sections within writing, not user interface components. User Interface components are covered under Success Criterion 4.1.2.
\end{itemize}
& & \\ \hhline{|===|}
\textbf{3.1.1 Language of Page:} The default human language of each Web page can be programmatically determined. (Level A)  & & \eoline
\textbf{3.2.1 On Focus:} When any component receives focus, it does not initiate a change of context. (Level A)  & & \eoline
\textbf{3.2.2 On Input:} Changing the setting of any user interface component does not automatically cause a change of context unless the user has been advised of the behavior before using the component. (Level A)  & & \eoline
\textbf{3.2.3 Consistent Navigation: }Navigational mechanisms that are repeated on multiple Web pages within a set of Web pages occur in the same relative order each time they are repeated, unless a change is initiated by the user. (Level AA)  & & \eoline
\textbf{3.2.4 Consistent Identification: }Components that have the same functionality within a set of Web pages are identified consistently. (Level AA)  & & \eoline
\textbf{3.2.5 Change on Request: }Changes of context are initiated only by user request or a mechanism is available to turn off such changes. (Level AAA) & & \eoline
\textbf{3.3 Input Assistance:} Help users avoid and correct mistakes.  &  & \\ \hhline{|===|}
\textbf{4.1.1 Parsing:} In content implemented using markup languages, elements have complete start and end tags, elements are nested according to their specifications, elements do not contain duplicate attributes, and any IDs are unique, except where the specifications allow these features. (Level A)  & & \eoline
\textbf{4.1.2 Name, Role, Value:} For all user interface components (including but not limited to: form elements, links and components generated by scripts), the name and role can be programmatically determined; states, properties, and values that can be set by the user can be programmatically set; and notification of changes to these items is available to user agents, including assistive technologies. (Level A)  & & \eoline
\end{longtable}