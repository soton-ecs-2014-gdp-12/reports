%% ----------------------------------------------------------------
%% Overlay.tex
%% ---------------------------------------------------------------- 
\chapter{Overlay - vgQuestions} \label{Chapter:Overlay}
The overlay was built up in stages:
\begin{enumerate}
\item Getting an overlay to appear at a set time
\item Adding in question types
\item Allowing custom question sets to be used
\item Returning the answers to a server
\end{enumerate}
\section{Question Sets} 
\label{Section:Question sets}
One of the main issues to address early on was how to represent the quiz and poll questions. The QTI (Question \& Test Interoperability) specification was investigated early on, but was found to be complicated and incomplete for the project needs. It was decided to write our own schema and use JavaScript files to store the information we needed. 

The schema is created to allow for configurability without the need for a large number of optional boolean operators. Thus we avoid having a ``shouldShowWhenPreviousQuestionIsAnsweredIncorrectly'' field on every questions as well as many other potential options. The main aim was to be expressive enough to allow all required situations without becoming too confusing. This was achieved with the use of inline functions within the JavaScript questions files. An example schema looks as follows:
\begin{lstlisting}[language=javascript]
{"first-question": {
  time: 1,
  items: [
    {
      id: "first",
      type: "single",
      question: "Single question example. Correct is the correct answer",
      options: [
        {
          name: "correct"
        },
        {
          name: "incorrect"
        }
      ],
      correctAnswer: "correct"
    },
    {
      id: "check-first",
      type: "single",
      question: "Answer incorrect, do you want to review the video",
      options: [
        {
          name: "Yes"
        },
        {
          name: "No"
        }
      ],
      action: function(questions, video) {
        if (questions.get("check-first").response === "Yes") {
          video.setTime(0);
        }
      },
      condition: function(questions, result) {
        return questions.get("first").isNotCorrect();
      }
    }
  ]
}}
\end{lstlisting}
Every question can have an action function along with a condition function. The action function is called when a question is answered. The condition function is called to determine which question is the question set to next show. Each function has a rich API passed along to it to allow complex actions to be performed.

The reason that this function-based approach could be taken was the use of HTML5 web workers. These are sandboxed separate threads that run in the background of webpages. They run independently of standard user-space scripts. More can be found out about their use within this project in Section BLAH. 

An early decision was to define the difference between a poll and a quiz question. It was decided that a poll is a type of quiz question that does not have a correct answer.

Initially basic question types (single choice, multiple choice and scale questions) were focused on. A variety of visualisations was implemented including check boxes, radio buttons and sliding scales. Validation was needed to ensure that the specified minimum and maximum limits were followed.

By having a standard schema that uses JavaScript functions it was possible to write template functions that could be outputted from the authoring tool and read by the overlay correctly.
\section{Front End} 
\label{Section:Front end}
The appearance of the overlay depends on the question type to be shown. The layout of these different types was carefully considered for accessibility and ease of use. Mockups were made and user studies were done in collaboration with a third year project student.

A set of example css files are supplied with the project that layout the questions according to the feedback received. Developers using any part of the project can use these styles as is, or modify/develop their own.
\section{Back End} 
\label{Section:Back end}
\subsection{Web Workers}
\label{Subsection:WebWorkers}
Web workers give a good way of running (sandboxed) background scripts that are computationally intensive. They are a way of multithreading - allowing multiple scripts to run simultaneously, avoiding the problem of unresponsive pages due to long running scripts. This is done by using message passing. 

For example, when an answer is submitted by the user this would sent a message to the webworker containing the answer. The webworker can then process this information without affecting the responsiveness of the page. Once the processing is complete the webworker can send a message back to the page to tell it what to do next.
