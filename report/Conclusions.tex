%% ----------------------------------------------------------------
%% Conclusions.tex
%% ---------------------------------------------------------------- 
\chapter{Conclusions} \label{Chapter: Conclusions}

\begin{preamble}
	The conclusions outlines the project and the degree of accomplishment focussing on customer satisfaction as the primary measure of success.
	\preamblequote{It was a `Jump to Conclusions' mat. You see, it would be this mat that you would put on the floor; and would have different conclusions written on it that you could jump to!}{Richard Riehle as Tom Smykowski in Office Space (1999)}
\end{preamble}

%comments are from mark scheme

% ++ Points ++
%
% Video Player
%  - Several improvements made to Videogular (focuisng on accessibility)
%  - Compatibility with a range of devices was reviewed (problems with iOS)
%
% Videogular Questions
%  - The Definition File can be created manually or using the authoring tool
%  - The styling can be changed using CSS
%
% Authoring Tool
%  - Produces the Definition File
%  - Highly accessible
%  - The standalone nature of the plugins has allowed for there easy use in the
%  authoring tool
%  - The authoring tool allows previewing, making it easy to use
%
% Analytics
%  - Plugin allows the events to be sent over the network and recorded
%
% Testing
%  - Emphasis on integration tests and deliverable reports
%  - Example sites
%  - Unit testing for several repositories
%  - analytics front end as an integration test
%  - load testing
%
% Future work
%  - Primarily to aid integration with the future version of Synote
%  - Fixing the YouTube plugin
%  - Mobile devices need more testing, problems are not always solvable
%  - Second screening
%  - Angular 1 support is likely to continue, so this is not an issue. Also
%  Angular is not key to the project, as there are lots of components that could
%  be extracted and used without Angular
%  - Video accessibility (captions and colour)?

\section{Introduction}



\section{Project Management}

During the process of the project a number of important papers and research books have been reviewed to provide the necessary background in eAssessment (\autoref{Chapter:Previous Work}). This project required a constant balance of academic and industrial work although the customer was interested in both. The industrial work involved producing the plugins to work with Videogular and a number of the deliverable examples. The acedemic work involved researching into into accessibility on browsers and operating systems to fulfil the customers requirements(\cref{Req:Browser compatability}, \cref{Req:OS compatability).

Throughout the project we performed work at a steady rate (\autoref{fig:tasksweek}) and work periods were distributed evenly over a 40 hour working week with some additional work at weekends and later at night (\autoref{fig:punch card}). The gantt charts produced (\autoref{App:Gantt Charts}) were used to plan progress during meeting to ensure that client priority deliverables had enough time.

%The agile process we followed allowed us to quickly do some things.

%Paragraph:
%planning and progress
%team worked independently
%optimal use of teams skills
%strengths weaknesses of team

%Paragraph:
%strengths weaknesses of process

%Paragraph:
%strengths weaknesses of design

%Paragraph:
%strengths weaknesses of results



%Extensive review of related work and references; References relate both to academic and industrial work
%Excellent planning and progress, team worked independently, optimal use of team’s skills

%Justified evaluation of the strengths and weaknesses of the team, process, design, and results

\section{Framework Design}

%Clear evidence of original thinking, considerable knowledge and understanding and ability to face considerable challenge

During the initial stages of the project, considerable effort was spent on the general approach. This did result in an approach that, while being quite complicated, drawing on a large number of tools and technologies, provided appropriate flexibility.

Using modern web technologies, e.g. the WebWorker (JavaScript sandbox) has allowed us to use a programmatic definition file, rather than using exhaustive data format.

During the course of the project, it was appropriate to separate the functionality that were implemented in to several small modules, which were then published.

Instantaneous feedback is key to learning, and this why this was built in as a key feature to the Videogular Questions plugin.

\section{Final conclusions}

Creating videogular-questions and the associated tools in the framework has been a challenging project. It now allows users to go right from the creation
stage, to the publishing stage, through to the analytics stage. Having met all goals the team looks forward to the client using it within Synote, and hopefully seeing it used elsewhere online.

All the components have been published, and there are good signs that these are already being used.

%Very challenging project, met all goals, ready to be used
%Publishable/patentable approach, thoroughly tested, evidence of customer satisfaction

