%% ----------------------------------------------------------------
%% Conclusions.tex
%% ---------------------------------------------------------------- 
\chapter{Conclusions} \label{Chapter: Conclusions}
It works.

\section{Future work}

Future work.

\subsection{Integration with Synote}

The most important part of our project is ensuring that the work we have done is easily integrated into Synote.

Therefore during the process we have been in communication with the current developer of Synote to ensure that the work we are doing will be able to be used.

One key requirement was that the work we are doing should be able to stand alone and not require any additional Synote framework but that Synote should be able to communicate with these pieces of work easily.

To ensure this all our services use an open standard to communicate using either REST calls, with an associated documentation of each call and how it can be used, or protocols such as web sockets. These services can be run as small webservers which will allow Synote to communicate with them easily. In addition since these are standalone we have been able to pick the best language for the associated service. This has decreased the complexity of our services which should reduce the time needed to learn how the software works in order to continue our work.

The core of the project uses AngularJS and the Videogular player as this was recommended as the latest version of Synote will be written in Angular. All of the plugins were written for the Videogular player and therefore will be able to be easily integrated into the primary Synote codebase. The plugins are the only part that will be integrated into Synote directly and therefore were best suited to be written in Angular. The other services will communicate with Synote and therefore did not need to be written in Angular.

\subsection{Mobile Interfaces - Second Screening}

A previous project [cite] worked on the possibility of creating a second screening application that worked with Synote. This allowed you to have two views of Synote and display different data on a second screen. This screen would usually be a mobile device for audience members [is this true?].

Creating interfaces that work well with mobile interfaces is a difficult task and therefore we have only spent the time to ensure that our basic examples work on mobile devices.

If we had time to continue our work we would look into extending the plugin that we have developed to have a "second screen" option for users. This would allow users to turn their mobile device into a second screen for our popups to allow them to answer the questions on their device. Further work would be concerned with getting the plugin to work on all devices as iPhones have shown to have significant issues with playing video and overlaying content (see section \ref{Section:Compatibility}).

This would also allow a user to take advantage of additional lecture features such as subtitling. Another project is currently being ran to get in lecture subtitles and this could be a feature added as one of our plugins.