\makeglossaries
\glossarystyle{list}
\newacronym[description={A defined interface that allows the access of features of a program or service}]{API}{API}{Application Programming Interface}
\newacronym[description={The definition file is used by vgQuestions plugin to create the overlaid quiz/poll}]{DF}{DF}{Definition File}
\newglossaryentry{vgQuestions}{text=vgQuestions,name=vgQuestions,description={A plugin to overlay quizzes and polls to videos},plural=vgQuestions}
\newacronym[description={Managed Learning Environment. The overall system used by an educational establishment to facilitate and manage learning}]{MLE}{MLE}{Managed Learning Environment}
\newacronym[description={Virtual Learning Environment. Component of a Managed Learning Environment responsible for the online interactions of students and tutors}]{VLE}{VLE}{Virtual Learning Environment}
\newglossaryentry{eAssessment}{text=eAssessment,name=eAssessment,description={The process of making, viewing and scoring assessments using a computer},plural=eAssessments}
\newacronym[description={Computer-Aided Assessment. \textit{see \gls{eAssessment}}}]{CAidA}{CAA}{Computer-Aided Assessment}
\newacronym[description={Computer Assisted Assessment. \textit{see \gls{eAssessment}}}]{CAssA}{CAA}{Computer Assisted Assessment}
\newacronym[description={Computer Based Assessment. \textit{see \gls{eAssessment}}}]{CBA}{CBA}{Computer Based Assessment}
\newglossaryentry{authoring}{text={Authoring Tool}, name={Authoring Tool}, description={Tool responsible for creating the questions and assembling the test}, plural={Authoring Tools}}
\newglossaryentry{delivery}{text={Assessment Delivery System}, name={Assessment Delivery System}, description={Tool responsible for displaying and scoring the assessments}, plural={Assessment Delivery Systems}}
\newacronym[description={Question and Test Interoperability. Quiz interoperability standard by IMS Global. \url{http://www.imsglobal.org/question/\#version2.1}}]{QTI}{QTI}{Question and Test Interoperability}
\newacronym[description={User Agent Accessibility Guidelines}]{UAAG}{UAAG}{User Agent Accessibility Guidelines}
\newacronym[description={Web Content Accessibility Guidelines: \url{http://www.w3.org/TR/WCAG20/} (Accessed: 15 Jan 15)}]{WCAG}{WCAG}{Web Content Accessibility Guidelines}
\newacronym[description={Web Accessibility Initiative}]{WAI}{WAI}{Web Accessibility Initiative}
\newacronym[description={Massive Open On-line Course. A free study course available over the internet}]{MOOC}{MOOC}{Massive Open On-line Course}
\newacronym[description={Model-View-Controller. An architectural pattern used in creating user interfaces}]{MVC}{MVC}{Model-View-Controller}
\newglossaryentry{AngularJS}{text={AngularJS},name={AngularJS},description={A client side Model-View-Controller architectural pattern},plural={AngularJS}}
\newglossaryentry{Videogular}{text={Videogular},name={Videogular},description={HTML5 video player for \gls{AngularJS}},plural={Videogular}}
\newacronym[description={Representational state transfer. An architectural pattern focusing on simple interfaces, scalability, portability, reliability and modifiability of components}]{REST}{REST}{Representational state transfer}
\newacronym[description={Accessible Rich Internet Applications. A Web Accessibility Initiative technical specification from the World Wide Web Consortium (W3C) that specifies how to make web pages more accessible}]{ARIA}{ARIA}{Accessible Rich Internet Applications}
\newacronym[description={Document Object Model. A convention for accessing and updating objects in HTML and XML documents. It is cross-platform and language-independent}]{DOM}{DOM}{Document Object Model}
\newglossaryentry{webworker}{text={WebWorker},name={WebWorker},description={A sandboxed separate thread that runs
in the background of the webpage}, plural={WebWorkers}}
\newacronym[description={Cascading Style Sheets. Language used for specifying the styling of a document written in a markup language}]{CSS}{CSS}{Cascading Style Sheets}
\newacronym[description={HyperText Markup Language version 5. A language used to describe Web pages, which has come to refer to the combination of HTML5, \gls{CSS} and JavaScript}]{HTML5}{HTML5}{HyperText Markup Language version 5}
