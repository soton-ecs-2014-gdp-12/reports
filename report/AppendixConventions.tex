\chapter{Coding Conventions} \label{App:Coding Conventions}

\begin{preamble}
	An appendix detailing the coding conventions and standards for the project.
	\preamblequote{How standards proliferate: \\ Situation: There are 14 competing standards \\ Researcher: 14?! Ridiculous! We need to develop one universal standard that covers everyone's use case! \\ Soon: \\ Situation: There are now 15 competing standards}{XKCD - web comic\footnote{XKCD Standards - \url{http://xkcd.com/927/}}}
\end{preamble}

\section{Introduction}

This appendix contains our coding conventions, from our GitHub repository
\footnote{\url{https://github.com/soton-ecs-2014-gdp-12/conventions}}.

\section{General}

File names should be in Unix form (lower-case, words separated by
hyphens, e.g. \texttt{a-directory/my-file.type}).

\section{CSS coding conventions}

Tabs should be used for indentation.

\subsection{Property ordering}

In general, properties further towards the top should affect layout,
while properties towards the bottom should affect appearance.

Specifically, the most common properties should be in this order:

\begin{lstlisting}
.class {
	position
	display

	flex /* etc. */

	top
	bottom
	left
	right
	z-index

	width  /* also min-, max- */
	height /* also min-, max- */

	padding
	border
	margin

	font /* etc. */
	text-align
	background /* etc. */
	color
}
\end{lstlisting}

\section{HTML coding conventions}

All full HTML pages should specify
\texttt{\textless{}!DOCTYPE html\textgreater{}}.

\subsection{Indentation}

Tabs should be used up to the indent level, with spaces for lining up
tags which break over multiple lines.

\subsection{Attributes}

Attribute values should be quoted.

The \texttt{id} attribute should always be first after the tag name,
followed by the \texttt{class} attribute. For meta tags, the
\texttt{name} should be specified first.

\subsection{Scripts}

Where possible, scripts should be imported at the end of the body. For
JavaScript files, the optional \texttt{type} attribute should be
omitted.

\section{Issue tracker usage guidelines}

If an issue doesn't seem to fit with any particular repository, file it
against the \texttt{videogular-questions} repository.

The issue type should be described with a label (one of bug,
enhancement, or investigate).

\subsection{waffle.io and Sprint management}

We have a waffle.io board
\footnote{\url{https://waffle.io/soton-ecs-2014-gdp-12/videogular-questions/}}
for tracking our Sprints.

Each Sprint will have a milestone on each repository, named ``Sprint '',
with the due date set to that of the Sprint Retrospective meeting. When
an issue is added to a Sprint, it should be added to that Sprint's
milestone.

\section{JavaScript coding
conventions}

Strict mode should be used (\texttt{'use strict';}). Semicolons should
never be omitted.

In object definitions, trailing commas should always be used. For
example:

\begin{lstlisting}[language=javascript]
obj = {
    foo: 'bar',
    baz: 'quux',
}
\end{lstlisting}

\subsection{Indentation}

Indentation should be done with tabs up to the indent level, and then
spaces for lining up multi-line statements. For example (where a
\texttt{\textgreater{}} is a tab and a \texttt{.} is a space):

\begin{lstlisting}[language=javascript]
function foo() {
>   if (bar === 4) {
>   >   baz("Some really ridiculously long string that should be "
>   >   ....+ "avoided, even mentioned in the coding conventions.");
>   }
}
\end{lstlisting}

This allows other developers to choose indent sizes without messing up
neatly lined-up parts.

\subsection{Naming}

Names should be camel case. The first letter should be lower-case,
except for class names or constructors:

\begin{lstlisting}[language=javascript]
LightBulb = (function() {
    function LightBulb() {
        // ...
    }

    return LightBulb;
})();

var numberOfEngineers = 5;
function changeLightBulb(bulb) {
    // ...
}
\end{lstlisting}

File names should still be in Unix form (e.g. \texttt{light-bulb.js}).
Unit test files should have the same name as the file which they test,
followed by \texttt{\_test} (e.g. \texttt{light-bulb\_test.js}).

\subsection{Types}

The section of JavaScript Garden on Types
\footnote{\url{https://bonsaiden.github.io/JavaScript-Garden/\#types}}
gives a number of guidelines (in the conclusion paragraphs) which should
be followed.

\subsection{JSHint}

JSHint directive comments should be kept to a minimum, with
configuration moved into the \texttt{.jshintrc} file where possible. If
file-specific configuration is necessary, it should go at the top of the
file (before \texttt{'use strict';}) unless:

\begin{itemize}
\item
  it is specific to a section of the file, or
\item
  it describes the changes made to the environment by a particular line
  of code, for example by the \texttt{importScripts} method in a Web
  Worker. In this case the directive should be on the next line,
  indented. For example:

\begin{lstlisting}[language=javascript]
importScripts("../../app/bower_components/videogular-questions/questions-worker.js");
    /* global loadAnnotations */
\end{lstlisting}
\end{itemize}

\subsection{AngularJS}

\subsubsection{Dependency injection}

When defining a directive, service, view, controller, etc., and
dynamically injecting dependencies, make sure to pass the parameter name
as a string into the array. For example:

\begin{lstlisting}[language=javascript]
angular.module("com.example.foobar", [])
    .directive(
    "ngFooBar",
    ["$window", "VG_STATES", function($window, VG_STATES) {
        ...
    }])
\end{lstlisting}

This stops minifiers from breaking the dependency information when they
rename parameters.

\section{LaTeX}

\subsection{TODOs}

For todo notes, use the \texttt{\textbackslash{}todo} command. For
example:

\begin{lstlisting}[language=tex]
\todo{Discuss farming in the Middle East.}
\end{lstlisting}

\subsection{References}

When making references, use \texttt{autoref}, like so:

\begin{lstlisting}[language=tex]
\autoref{my-label}
\end{lstlisting}

\subsection{Labelling scheme}

Figure labels should begin with \texttt{Figure:}, chapter labels with
\texttt{Chapter:}, and section labels with \texttt{Section:}.

\section{Python coding conventions}

PEP8 \footnote{\url{http://legacy.python.org/dev/peps/pep-0008/}} should be
followed.

\subsection{File names}

File names should be Python style, that is, words separated by
underscores (e.g. \texttt{voluminous\_octopus.py}).

When a virtual environment is used, it should be called \texttt{venv}.
Any dependencies should be defined in \texttt{requirements.txt}.
