%% ----------------------------------------------------------------
%% ProjectManagement.tex
%% ---------------------------------------------------------------- 
\chapter{Project Management} 
\label{Chapter: Project Management}
This project was run using a SCRUM project management approach. This involved having weekly sprints where each person completed a number of task, either alone or collaborating with others. Planning meetings were held at the start of each sprint where the issues in the backlog were discussed. They were prioritised and given a point value which related to how much time and effort the issue would take to resolve. The highest priority issues were then assigned. Each person had a maximum capacity of points for each week. Additionally retrospectives were held just prior to the planning meetings to discuss the previous sprints and the way in which the project management was being carried out.

Throughout the project we were careful to adapt when needed, and not be too constrained by the SCRUM model. At points in the project where progress presentations had to be created, practised and presented we ran ``mini-sprints'' both for the presentation, and work to be completed in the remainder of the week.
%SCRUM roles
%Burndown charts and other statistics

To allow for collaborative work a GitHub organisation\footnote{\url{https://github.com/soton-ecs-2014-gdp-12/}} was used. This acted as our central source code repository for not only all directly related work, but also work on presentations and reports. Separate repositories for each area of the project are located here for ease of access. Using GitHub made it easier to interact with other open source projects, and all our pull requests and releases were directed from here. 

GitHub also provided an issue management system on a repository by repository basis. This allowed us to track conversations around different bugs and enhancements as the project progressed.
The group used \url{http://www.waffle.io} to manage the issues backlog from a project wide perspective as this would interface directly with GitHub.