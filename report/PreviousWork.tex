%% ----------------------------------------------------------------
%% PreviousWork.tex
%% ---------------------------------------------------------------- 
\chapter{Previous Work} 
\label{Chapter:Previous Work}
\section{Academic Work} 
\label{Section:Academic Work}
A Managed Learning Environment (MLE) is the overall system used by an educational establishment to facilitate and manage learning. This will be comprised of several subsystems (as seen in \autoref{Figure: MLE}).

\begin{figure}[h]
	\centering 
		\includegraphics[scale=0.4]{../figures/MLE.png} 		
	\caption{\label{Figure: MLE} Components of an MLE \citep{mle}} 	
\end{figure}

This project focuses on the Assessment system in the Virtual Learning Environment (VLE).

\subsection{eAssessment}
An eAssessment (also known as Computer-Aided Assessment (CAA), Computer Assisted Assessment (CAA) or Computer Based Assessment (CBA)) is the process of making, viewing and scoring assessments using a computer. In this report these processes will be carried out by several tools - the ``Authoring Tool" will be responsible for creating the questions and assembling the test and the ``Assessment Delivery System" will be responsible for displaying and scoring the assessments. These tools must be interoperable, meaning they must use data formats that are compatible with each other.

This project focuses on the use of quizzes and polls within videos as a means of assessment.

\subsubsection{Quiz Definition Standards}
\label{Subsubsection:Quiz Definition Standards}
In order to store and display the questions in an eAssessment, a schema for the questions must be defined. The only formally defined standard found was IMS Global's Question and Test Interoperability (QTI) specification\footnote{\url{http://www.imsglobal.org/question/\#version2.1} Last Accessed: 24 Nov 14}. Some other quiz (wiki-like) definitions were found including Moodle's General Import Format Technology (GIFT)\footnote{\url{https://docs.moodle.org/28/en/GIFT_format} Last Accessed: 24 Nov 14} and Aiken formats\footnote{\url{https://docs.moodle.org/23/en/Aiken_format} Last Accessed: 24 Nov 14}.

GIFT is a question mark up by Moodle\footnote{\url{https://moodle.org} Last Accessed: 24 Nov 14}. It allows users to use a text editor to create questions in some simple formats. There is a strict compliance to specific syntax required and for complex questions it is no longer intuitive \citep{failQTI}. Aiken is another mark up that is designed to be more easily human-readable, again it has strict syntax that must be adhered to or the imports to Moodle would fail.

QTI is an interoperability standard for quizzes. \cite{wikieassessment} state:
\begin{quote}
The major promise of QTI is that by introduction of common format developers can concentrate on developing innovative tools, whereas teachers can focus on defining new and groundbreaking methods of how to apply those tools in an online environment.
\end{quote}
It is a very complex specification with many ambiguous or optional elements. The complexity of the specification increases the likelihood for errors in the implementation as it opens up opportunities for developers to interpret the specification in different ways \citep{failQTI}. This means the intent of the original author may be lost.

The QTI Specification is ``designed to facilitate interoperability between a number of systems". The Overview \citep{qtiOverview} splits the eAssessment sytem into five systems: authoring tool, item bank, test construction tool, assessment delivery system and learning system (see \autoref{Figure: QTI components}). 

In the QTI definition the authoring tool is used to create individual questions. However many systems combine this with the Item Bank and Test Construction Tool. For this report ``Authoring Tool" refers to this combination and, as such, an authoring tool is used to create whole quizzes.

\begin{figure}[h]
	\centering 
		\includegraphics[scale=0.5]{../figures/componentsQTI.png} 		
	\caption{\label{Figure: QTI components} QTI components diagram \citep{qtiOverview}} 	
\end{figure}

QTI v2.1 was finalised in 2012 \citep{qtiOverview} but up to this point the standard was not widely used \citep{eps265979} as there were several fundamental flaws in the standard. When the v2.1 draft was withdrawn in 2009 Rib Abel (IMS GLC CEO) was quoted as saying (on v2.0) ``it’s deficiencies are well known and IMS does not recommend implementation of it\dots the only version of QTI that is fully endorsed by IMS GLC is v1.2.1"\footnote{\url{http://lists.ucles.org.uk/public/ims-qti/2009-March/001463.html}}. This standard gave definitions in natural language. These could often be long-winded and ambiguous, making the standard difficult to implement\citep{failQTI, Sclater2007}.

% Templated (change the numbers) or adaptive questions (branching questions)

\subsubsection{Usage of Video in eAssessment}
\label{Subsubsection:Usage of Video in eAssessment}
Video is often used as a medium to convey educational material as it appeals to different learning styles. When video is streamed there is a lack of interactivity and user control \citep{eps267281}. To involve the user more, interactive elements must be added. \cite{eps267281} found that 75\% of users agreed or strongly agreed that interactive video had enhanced their learning experience but at the time of the paper there was no evidence of interactive video being used as a learning tool.

%Feedback - meaningful, progress in understanding \citep{eps265979}
\subsubsection{Accessibility and Usability}
\label{Subsubsection:Accessibility and Usability}
To allow interoperability between the eAssessment systems the QTI specification involves both the model and the view of the data \citep{wikieassessment} making accessibility difficult to add in later. This means many systems that comply with the QTI specification are inaccessible. Moodle, for example, uses a video player to display content that is completely inaccessible to a user only using a keyboard\footnote{\url{https://tracker.moodle.org/browse/MDL-36081} Last Accessed: 24 Nov 14}.

eAssessment systems require an authoring tool that is easily understood by people who do not have an in depth technical knowledge (e.g. teachers). Often authoring tools that implement the QTI standard are too technical for this \citep{wikieassessment}. These tools need to improve their accessibility and usability, keeping the time commitments for users reasonable \citep{eps271236, eps265979} or they will not be used.

There are many ways of conveying content in a video and none of the methods alone will generally convey the whole message. Most video searches stay on a whole resource level as there is a lack of semantic interlinking \citep{eps273063}. Annotation systems such as Synote\footnote{\url{http://synote.org/synote/} Last Accessed: 24 Nov 14} aim to make videos accessible by adding transcripts and other annotations to them. The videos being annotated may not be owned by the annotator.

\subsubsection{Analytics}
\label{Subsubsection:Analytics}



\newif\ifnote
\notefalse
\ifnote
In \cite{eps265979}:
R2Q2 (deprecated - replaced by QTI Engine)
Constructr - constructr.qtitools.org - proof-of-concept
Playr - playr.qtitools.org - deprecated - effectively our overlays
\fi

\section{Industrial Work} 
\label{Section:Industrial Work}