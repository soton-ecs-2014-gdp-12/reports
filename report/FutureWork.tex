%% ----------------------------------------------------------------
%% FutureWork.tex
%% ---------------------------------------------------------------- 
\chapter{Future work} \label{Chapter: Future Work}

\chapterpreamble{This framework is very much just the beginning, there are a number of interesting improvements that can be made to enhance and extend the project. This chapter details a number of these potential improvements.}

\section{Integration with Synote}

The most important part of our project is ensuring that the work we have done is easily integrated into Synote.

Therefore during the process we have been in communication with the current developer of Synote to ensure that the work we are doing will be able to be used.

One key requirement was that the work we are doing should be able to stand alone and not require any additional Synote framework but that Synote should be able to communicate with these pieces of work easily (see \cref{Req:Standalone}).

To ensure this all, our services use an open standard to communicate using either \gls{REST} calls, with an associated documentation of each call and how it can be used, or protocols such as web sockets. These services can be run as small webservers which will allow Synote to communicate with them easily. In addition since these are standalone we have been able to pick the best language for the associated service. This has decreased the complexity of our services which should reduce the time needed to learn how the software works in order to continue our work.

The core of the project uses \gls{AngularJS} and the \gls{Videogular} player as this was recommended as the latest version of Synote will be written in Angular. All of the plugins were written for the \gls{Videogular} player and therefore will be able to be easily integrated into the primary Synote codebase. The plugins are the only part that will be integrated into Synote directly and therefore were best suited to be written in Angular. The other services will communicate with Synote and therefore did not need to be written in Angular.

\section{Additional Video Players}

Since our plugins are built upon the \gls{Videogular} system we have not been concerned with how the video is played. The base \gls{Videogular} system plays the common video formats.

To further expand our system we could look at fixing the \gls{Videogular} YouTube Plugin\footnote{\url{https://github.com/NamPNQ/bower-videogular-youtube}}. This would allow users to specify a youtube video to use for the quiz and removes the problems with getting all formats of video for all browsers.

\section{Video Encoding}

\todo{Talk about how we could produce something that removes this requirement}

\section{Mobile Interfaces}

Creating interfaces that work well with mobile interfaces is a difficult task and therefore we have only spent the time to ensure that our basic examples work on most mobile devices.

Further work would be concerned with getting the plugin to work on all devices as iPhones have shown to have significant issues with playing video and overlaying content (see \autoref{Section:Compatibility}).

\section{Second Screen}

A previous project \todo{cite} worked on the possibility of creating a second screening application that worked with Synote. This allowed you to have two views of Synote and display different data on a second screen. This screen would usually be a mobile device for audience members \todo{is this true?}.

If we had time to continue our work we would look into extending the plugin that we have developed to have a "second screen" option for users. This would allow users to turn their mobile device into a second screen for our popups to allow them to answer the questions on their device.

This would also allow a user to take advantage of additional lecture features such as subtitling. Another project is currently being ran to get in lecture subtitles and this could be a feature added as one of our plugins.

\section{Angular 2}

At the 2014 ng-europe conference during a talk about the future development of Angular details about the 2.0 release were given. It is slated for an early 2016 release, with Angular 1.x receiving bug fixes for anther 2 years.

Currently it seems that Angular 2.0 is a complete rewrite of the Angular ecosystem and is only related to Angular 1.x by name. Upon its release applications wishing to stay upto date will need to be rewritten from scratch. As this project is currently built around Videogular, it would first need to be updated and rewritten. This framework's main customer, Synote, is currently in development and as such is using Angular 1.x. 

As such there is potential future work for investigating Angular 2.0 upon its final release. At that time much more information regarding potential upgrade paths would exist.

\section{Cuepoints and Heatmap accessibility}
Currently the only way of getting information from the \gls{Videogular} Cuepoints and \gls{Videogular} Heatmap plugins is through the use of colour. Although these colours are customisable these plugins are not very accessible. An improvement that could be made in future is to have information available when the marked section is in focus (on hover and when it is playing) that gives the details used by the plugin. For example an area of a heat map could display the number of times the section has been viewed, and the start and end times of the section.

\section{Conclusions}