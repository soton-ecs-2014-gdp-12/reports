\chapter{Minutes of Meetings} \label{Chapter:Minutes of Meetings}

This appendix contains the minutes of meetings held with the group's supervisor, client and second examiner, as recorded by Harry Cutts. It also contains minutes for our Sprint Retrospectives, where there were any.

Unless otherwise stated, all five group members attended all meetings. Mike Wald attended all client/supervisor meetings as both client and supervisor.

\section{29th September 2014}\label{Minutes:2014-09-29}

\subsection{Client/supervisor meeting}

Yunjia Li (lead developer on Synote) also attended.

\subsubsection{History of Synote}

Mike Wald started by giving a brief history of Synote.

The 2008 version (which is the version running on \texttt{synote.org})
was built with Google Web Toolkit and Grails.

The 2010 version replaced the front-end, but re-used the back-end. A
mobile interface was later added to this version.

A version for the Raspberry Pi was produced.

The new version of Synote is to be built with a node.js back-end, which
will interact with an AngularJS front-end via a ReST API. The code is to
be hosted on GitHub.

\subsubsection{Our task}

Our task is to work on software for interactive video quizzes to be
integrated with Synote at a later date. The solution must be accessible.
We may wish to use the QTI standard for defining quizzes.

This has been done before by a Masters student called Nadia, but her
work was not compatible with AngularJS. Mike will send Nadia's
implementation and report to us. There is also an existing AngularJS
library for interactive video quizzes, which we could fork.

Other requirements:

\begin{itemize}
\itemsep1pt\parskip0pt\parsep0pt
\item
  Use ReSTful APIs to connect the front-end and back-end
\item
  Define a data structure which we can statically serve for the proof of
  concept
\item
  Consider mobile browsers
\item
  Make proper tests
\item
  We'll need to choose a framework
\item
  Later in the project, create a quiz authoring tool
\end{itemize}

The project should be managed in an Agile way. Harry suggested that we
follow SCRUM. Yunjia recommended the RallyDev application for issue
management.

We will hold these meetings with Mike regularly at 14:00 on Mondays. The
group will meet after the GDP briefing on Wednesday.

\subsection{After-meeting}

Sam suggests that we create a very basic prototype and then ask whether
this is what they want.

Harry suggested not scheduling anything over Christmas.

We should find some users and talk to them

SCRUM (with 1 week sprints) would be good to try.

We each should:

\begin{itemize}
\itemsep1pt\parskip0pt\parsep0pt
\item
  Read Nadia's report
\item
  Have a look at the AngularJS Video Quiz module (Chris will email us the link)
\end{itemize}

Harry will create the Google Drive folder, upload the minutes, create a
GitHub organisation and add everyone in the group.

\section{6th October 2014}\label{Minutes:2014-10-06}

\subsection{Client/supervisor meeting}

Shameem Bajar, a third-year student of Information Technology in
Organisations, also attended.

We presented a prototype system built on Videogular over the last week.

Videogular Quiz wasn't worth building on.

Shameem is doing a third-year project, and will be interacting with
users.

Aspects which we haven't covered in our prototype:

\begin{itemize}
\itemsep1pt\parskip0pt\parsep0pt
\item
  Jumping back to sections when questions are answered incorrectly
\item
  Analytics on user behaviour around the quiz
\end{itemize}

A big use case for this project is for the live version. Another is for
MOOCs.

Our technical goals are to overlay quiz questions and polls over online
videos, with analytics support.

Over the next week, we will:

\begin{itemize}
\itemsep1pt\parskip0pt\parsep0pt
\item
  write the brief (sending a draft to Mike),
\item
  make the prototype code more robust, and
\item
  test the prototype on multiple platforms, building up a list of
  issues.
\end{itemize}

As soon as possible, we will get a demo working which can be used by
Shameem to get user feedback and stories.

\subsection{After-meeting}

We could draw some state machines of possible quizzes.

Some questions for us to answer:

\begin{itemize}
\itemsep1pt\parskip0pt\parsep0pt
\item
  Would MediaElement.js be worth investigating?
\item
  Is Videogular's accessibility up to Mike's standards?
\end{itemize}

\section{13th October 2014}\label{Minutes:2014-10-13}

\subsection{Client/supervisor meeting}

We demonstrated our progress on the prototype, including it running on a
phone and a tablet. Mike reiterated that the user should be able to jump
back in the video when a question is answered incorrectly.

We asked how we are to acquire realistic data with which to test the
analytics system. Mike rejected our selection of him using a prototype
system for a lecture, but suggested that Shameem could use focus groups
to acquire data.

He also clarified that the actual analysis of the data is basically as a
proof-of-concept. The important part is to have the front-end collect
the data (in the form of event logs).

We attempted to demonstrate some accessibility improvements which we
have made and contributed to Videogular, but they had not yet been
deployed on the demonstration machine.

We asked whether we could access the code for the new version of Synote,
and learned that Yunjia hasn't started on the new Synote yet.

\section{20th October 2014}\label{Minutes:2014-10-20}

\subsection{Client/supervisor meeting}

We demonstrated:

\begin{itemize}
\itemsep1pt\parskip0pt\parsep0pt
\item
  Videogular accessibility improvements
\item
  Skipping back in the video when a wrong answer is given
\item
  The new stars question type
\end{itemize}

We explained the architecture on which we had decided for the questions
front-end, in which a WebWorker (a sandboxed JavaScript thread) is used
to execute potentially unsafe quiz definitions. This allows us to use
JavaScript to define quizzes in a flexible manner without a complicated
markup language.

Mike said he would think about organising a focus group to collect data
from.

We reported on our first sprint of the project.

We said we would prepare a stable demo for Wednesday, and give the link
to Shameem for user feedback.

\subsubsection{The presentation}

We discussed how to pitch the idea of our project in the up-coming
progress presentation.

Chris Hewett suggested we `sell' the idea of teaching in small chunks
and testing on each individual chunk. Small chucks can be revisited
easily, but splitting a video into those small chunks is very
time-consuming. Instead, our project will allow questions to be inserted
at the end of each chunk in a longer video.

\subsection{Sprint retrospective}

In the traditional SCRUM way, we discussed how well our first sprint had
gone.

\subsubsection{The good}

\begin{itemize}
\itemsep1pt\parskip0pt\parsep0pt
\item
  We got everything that was assigned in the planning meeting done.
\item
  We communicated well.
\end{itemize}

\subsubsection{The bad}

\begin{itemize}
\itemsep1pt\parskip0pt\parsep0pt
\item
  We underestimated the amount of work we could get done.
\item
  Our points allocations were out of whack quite a bit.
\end{itemize}

\section{27th October 2014}\label{Minutes:2014-10-27}

There was no meeting this week as Mike was away.

\subsection{Retrospective}

We didn't all make our points targets this week. This was mostly due to
the presentation.

Poll and Cuepoints issues weren't well defined enough.

Chewett is blocking on lack of documentation and schemas when creating
examples. We should create issues to finalise schemas.

\section{10th November 2014}\label{Minutes:2014-11-10}

\subsection{Client/supervisor meeting}

Yunjia Li and Shameem Bajar also attended.

We demonstrated:

\begin{itemize}
\itemsep1pt\parskip0pt\parsep0pt
\item
  Charts showing results of in-video polls.
\item
  The heat map plug-in for Videogular.
\item
  Our Videogular Analytics plug-in acquiring analytics data, and it
  being processed into a heat map.
\end{itemize}

We mentioned that we need to use HTML5 events (including seek) instead
of watches to detect changes in the video state. We also need to display
the heat map data using the plug-in we've created for Videogular.

Yunjia raised concerns about the perceived difficulty of deploying our
server-side code. We said we would write a document describing the
deployment process to an Apache server.

\subsubsection{Shameem's study}

Shameem presented some initial results of her user studies. The
presentation will be sent to us shortly.

Feedback on the idea was positive. No users mentioned mobile devices,
except that a participant would submit poll responses from a phone
during a lecture (which is out of the scope of our project).

Many suggestions were made by users, some of which were out of scope.
The following were in scope:

\begin{itemize}
\itemsep1pt\parskip0pt\parsep0pt
\item
  Add number of attempts field to authoring tool.
\item
  Give a grade at the end of the quiz.
\item
  Recommend further videos on the same topic after the video is finished
  (which we could display as custom HTML on a results page).
\end{itemize}

We discussed the upcoming presentation, during which we are planning to
demonstrate the analytics view.

\subsection{Retrospective}

\begin{itemize}
\itemsep1pt\parskip0pt\parsep0pt
\item
  We have inconsistency between tabs and spaces for indentation in some
  files.
\item
  Some external libraries are stored inconsistently (e.g.~Bootstrap is a
  submodule in the analytics back-end, d3 is a file in the repository).
\item
  We can use \texttt{bower link} instead of the shell script for using
  Bower with modules which are in development. This will be done in the
  next sprint.
\end{itemize}

\section{17th November 2014}\label{Minutes:2014-11-17}

\subsection{Sprint retrospective}

\begin{itemize}
\itemsep1pt\parskip0pt\parsep0pt
\item
  Number of points per person was good
\end{itemize}

\subsection{Client/supervisor meeting}

We demonstrated the authoring tool UI which has been designed, but not
made functional.

We showed the presentation planned for Wednesday. Mike suggested to
mention the interactive features such as skipping back in the video.

\section{24th November 2014}\label{Minutes:2014-11-24}

\subsection{Meeting with Gary Wills}

\begin{itemize}
\item
  Report is what Gary's going to mark
\item
  UML in design section
\item
  Why didn't we present it to users for evaluation?

  \begin{itemize}
  \itemsep1pt\parskip0pt\parsep0pt
  \item
    The customer (Mike) thinks that it is not viable.
  \end{itemize}
\item
  Write about testing and/or scenario based testing instead.
\item
  Could evaluate based on ``How well does this output suit your
  requirements?''
\item
  Doing team skills audit will improve mark
\item
  Don't include `if we have time' things in goals
\item
  Should have a consistent voice throughout the report
\item
  We \textbf{should not} be working through Christmas
\item
  Manage the customer (Mike)
\item
  ``The key to good assessment is instant feedback.''
\item
  Go through report with fine-tooth spelling and grammar comb before
  submitting report.
\end{itemize}

\subsection{Client/supervisor meeting}

Mike said he was generally happy with our latest progress presentation.

Looking at the authoring tool:

\begin{itemize}
\itemsep1pt\parskip0pt\parsep0pt
\item
  It would be good to make single choice question type a special case of
  multiple choice question.
\item
  A ``Load current time'' button next to the time selector would be
  useful.
\end{itemize}

We should mention that (some) server implementations are example only
and have no scalability guarantees.

\section{1st December 2014}\label{Minutes:2014-12-01}

\subsection{Client/supervisor meeting}

Yunjia Li also attended.

We demonstrated the authoring tool.

Yunjia suggested something for the Further Work report section: making
encoding the video for different browsers/platforms more user-friendly.

Chewett proposed a list of deliverables:

\begin{itemize}
\itemsep1pt\parskip0pt\parsep0pt
\item
  Videogular Questions

  \begin{itemize}
  \itemsep1pt\parskip0pt\parsep0pt
  \item
    Example proof-of-concept site (Videogular Questions Example)
  \end{itemize}
\item
  Videogular Cuepoints

  \begin{itemize}
  \itemsep1pt\parskip0pt\parsep0pt
  \item
    As demonstrated is Videogular Questions Example
  \end{itemize}
\item
  Videogular Heat Maps
\item
  Videogular Analytics

  \begin{itemize}
  \itemsep1pt\parskip0pt\parsep0pt
  \item
    API specification for Videogular Analytics
  \item
    Analytics back-end is an example only
  \end{itemize}
\item
  Authoring tool
\end{itemize}

Mobile browsing will be part of future work.

Mike approved that list of deliverables.

\section{8th December 2014}\label{Minutes:2014-12-08}

\subsection{Client/supervisor meeting}

Yunjia Li also attended.

\subsubsection{Sign off on deliverables}

The deliverables had been sent to Mike last Thursday. He had found that
the heat map did not appear in the analytics interface in Google Chrome.
We clarified a misunderstanding about the vertical scales on results
charts.

Yunjia asked about the status of the seek event in the analytics
plug-in, which has not been implemented but should not be problematic.

Mike and Yunjia said that they were satisfied with the deliverables.
Yunjia said they were `pretty cool'.

Mike said he'd be happy to comment on draft report stuff we send him.

\subsection{After-meeting}

We discussed the report structure, and decided to put all the testing in
one section (with subsections for each component), to emphasize the
range of testing methods we used.
