\documentclass[12pt,a4paper]{article}
\usepackage{pgfgantt}
\usepackage{pdflscape}
\usepackage{url}
\usepackage{standalone}
\usepackage[hidelinks]{hyperref}
\usepackage[intoc]{nomencl}
\usepackage{caption}
\usepackage{parskip}

\usepackage{pdfpages}
\usepackage{todonotes}

\usepackage[top=2.4cm, bottom=2.4cm, inner=3.5cm, outer=2.4cm]{geometry}

\usepackage{microtype}

\begin{document}
\pagenumbering{gobble}

\begin{Large}
\textbf{School of Electronics And Computer Science
}\end{Large}\\

\begin{Large}
\textbf{ELEC6200    MEng Group Design Project
}\end{Large}\\

\begin{Large}
Project Specification And Plan
\end{Large}\\\\

\textbf{Title:}
Interactive Web video quizzes and polls\\

\textbf{Supervisor:}	
Mike Wald\\

\textbf{Team Members:}\\
Christopher Baines\\
Samuel Bennett\\
Harry Cutts\\
Christopher Hewett\\
Maria Lynch\\

\textbf{Customer:}
Mike Wald\\

\textbf{Project Specification:}
A library will be implemented which allows quizzes and polls to be overlaid on Web videos, playable in the main browsers (Firefox, IE and Chrome) on different platforms (PC, Mac and Android). These videos could be any of the main video formats (MP4, WebM, OGG). Questions will be specified in JSON files interpreted by the library. Work will be done to create an authoring tool for the system. Accessibility considerations for the overlaid videos and authoring tool will be investigated. Metrics of user behaviour will be recorded and methods of displaying these to the author explored.
\\

\textbf{Work plan:}
Gantt charts can be found in figures~\ref{ganttChart1} and \ref{ganttChart2}.

SCRUM, an Agile methodology, will be followed. This means that the work done in each Sprint will be subject to change until the end of the Sprint Planning Meeting on the Monday of each week.

The aim for the end of Sprint 3 will be to have the quiz and poll overlay complete. During the following six Sprints the authoring and analytics tools will be created in parallel.

Please note that each sprint contains some time in which report writing will be completed.

\begin{landscape}

\ganttset{%
    calendar week text={%
        W~\currentweek%
    }%
}

\begin{figure}[h!]
\begin{ganttchart}[
    hgrid,
    vgrid,
    x unit=1.8mm,
    time slot format=isodate
    ]{2014-09-29}{2014-12-14}
    \gantttitlecalendar{year, month, week=1} \\


    \ganttbar{Prototyping}{2014-09-29}{2014-10-12} \\
    \ganttgroup{Weekly Sprints}{2014-10-13}{2014-12-14} \\
    \ganttbar{Quiz and poll display component}{2014-10-13}{2014-11-02}\\
    \ganttmilestone{Progress Seminar 1}{2014-10-22} \\
    \ganttbar{Authoring tool}{2014-11-03}{2014-12-14}\\
    \ganttbar{Analytics tool}{2014-11-03}{2014-12-14}\\
    \ganttmilestone{Progress Seminar 2}{2014-11-19}
    
\end{ganttchart}
\caption{Gantt Chart prior to the Christmas vacation.}
\label{ganttChart1}
\end{figure}

\begin{figure}[h!]
\begin{ganttchart}[
    hgrid,
    vgrid,
    x unit=1.8mm,
    time slot format=isodate
    ]{2014-12-15}{2015-02-22}
    \gantttitlecalendar{year, month, week=12} \\

    \ganttgroup{Christmas}{2014-12-15}{2015-01-04} \\
    \ganttgroup{Exams}{2015-01-12}{2015-01-24} \\
    \ganttbar{Group report finalisation and review}{2015-01-05}{2015-01-29}\\
    \ganttmilestone{Group Report}{2015-01-29} \\
    \ganttbar{Individual reflection}{2015-01-25}{2015-02-02}\\
    \ganttmilestone{Individual Reflection submission}{2015-02-02} \\
    \ganttbar{Poster and Presentation creation}{2015-01-30}{2015-02-05}\\
    \ganttmilestone{Poster and Presentation submission}{2015-02-05} \\
    \ganttbar{Presentation rehearsal}{2015-02-03}{2015-02-11}\\
    \ganttmilestone{Final Presentation}{2015-02-11}
\end{ganttchart}
\caption{Gantt Chart after the  Christmas vacation}
\label{ganttChart2}
\end{figure}

\end{landscape}

\end{document}
